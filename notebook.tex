
% Default to the notebook output style

    


% Inherit from the specified cell style.




    
\documentclass[11pt]{article}

    
    
    \usepackage[T1]{fontenc}
    % Nicer default font (+ math font) than Computer Modern for most use cases
    \usepackage{mathpazo}

    % Basic figure setup, for now with no caption control since it's done
    % automatically by Pandoc (which extracts ![](path) syntax from Markdown).
    \usepackage{graphicx}
    % We will generate all images so they have a width \maxwidth. This means
    % that they will get their normal width if they fit onto the page, but
    % are scaled down if they would overflow the margins.
    \makeatletter
    \def\maxwidth{\ifdim\Gin@nat@width>\linewidth\linewidth
    \else\Gin@nat@width\fi}
    \makeatother
    \let\Oldincludegraphics\includegraphics
    % Set max figure width to be 80% of text width, for now hardcoded.
    \renewcommand{\includegraphics}[1]{\Oldincludegraphics[width=.8\maxwidth]{#1}}
    % Ensure that by default, figures have no caption (until we provide a
    % proper Figure object with a Caption API and a way to capture that
    % in the conversion process - todo).
    \usepackage{caption}
    \DeclareCaptionLabelFormat{nolabel}{}
    \captionsetup{labelformat=nolabel}

    \usepackage{adjustbox} % Used to constrain images to a maximum size 
    \usepackage{xcolor} % Allow colors to be defined
    \usepackage{enumerate} % Needed for markdown enumerations to work
    \usepackage{geometry} % Used to adjust the document margins
    \usepackage{amsmath} % Equations
    \usepackage{amssymb} % Equations
    \usepackage{textcomp} % defines textquotesingle
    % Hack from http://tex.stackexchange.com/a/47451/13684:
    \AtBeginDocument{%
        \def\PYZsq{\textquotesingle}% Upright quotes in Pygmentized code
    }
    \usepackage{upquote} % Upright quotes for verbatim code
    \usepackage{eurosym} % defines \euro
    \usepackage[mathletters]{ucs} % Extended unicode (utf-8) support
    \usepackage[utf8x]{inputenc} % Allow utf-8 characters in the tex document
    \usepackage{fancyvrb} % verbatim replacement that allows latex
    \usepackage{grffile} % extends the file name processing of package graphics 
                         % to support a larger range 
    % The hyperref package gives us a pdf with properly built
    % internal navigation ('pdf bookmarks' for the table of contents,
    % internal cross-reference links, web links for URLs, etc.)
    \usepackage{hyperref}
    \usepackage{longtable} % longtable support required by pandoc >1.10
    \usepackage{booktabs}  % table support for pandoc > 1.12.2
    \usepackage[inline]{enumitem} % IRkernel/repr support (it uses the enumerate* environment)
    \usepackage[normalem]{ulem} % ulem is needed to support strikethroughs (\sout)
                                % normalem makes italics be italics, not underlines
    

    
    
    % Colors for the hyperref package
    \definecolor{urlcolor}{rgb}{0,.145,.698}
    \definecolor{linkcolor}{rgb}{.71,0.21,0.01}
    \definecolor{citecolor}{rgb}{.12,.54,.11}

    % ANSI colors
    \definecolor{ansi-black}{HTML}{3E424D}
    \definecolor{ansi-black-intense}{HTML}{282C36}
    \definecolor{ansi-red}{HTML}{E75C58}
    \definecolor{ansi-red-intense}{HTML}{B22B31}
    \definecolor{ansi-green}{HTML}{00A250}
    \definecolor{ansi-green-intense}{HTML}{007427}
    \definecolor{ansi-yellow}{HTML}{DDB62B}
    \definecolor{ansi-yellow-intense}{HTML}{B27D12}
    \definecolor{ansi-blue}{HTML}{208FFB}
    \definecolor{ansi-blue-intense}{HTML}{0065CA}
    \definecolor{ansi-magenta}{HTML}{D160C4}
    \definecolor{ansi-magenta-intense}{HTML}{A03196}
    \definecolor{ansi-cyan}{HTML}{60C6C8}
    \definecolor{ansi-cyan-intense}{HTML}{258F8F}
    \definecolor{ansi-white}{HTML}{C5C1B4}
    \definecolor{ansi-white-intense}{HTML}{A1A6B2}

    % commands and environments needed by pandoc snippets
    % extracted from the output of `pandoc -s`
    \providecommand{\tightlist}{%
      \setlength{\itemsep}{0pt}\setlength{\parskip}{0pt}}
    \DefineVerbatimEnvironment{Highlighting}{Verbatim}{commandchars=\\\{\}}
    % Add ',fontsize=\small' for more characters per line
    \newenvironment{Shaded}{}{}
    \newcommand{\KeywordTok}[1]{\textcolor[rgb]{0.00,0.44,0.13}{\textbf{{#1}}}}
    \newcommand{\DataTypeTok}[1]{\textcolor[rgb]{0.56,0.13,0.00}{{#1}}}
    \newcommand{\DecValTok}[1]{\textcolor[rgb]{0.25,0.63,0.44}{{#1}}}
    \newcommand{\BaseNTok}[1]{\textcolor[rgb]{0.25,0.63,0.44}{{#1}}}
    \newcommand{\FloatTok}[1]{\textcolor[rgb]{0.25,0.63,0.44}{{#1}}}
    \newcommand{\CharTok}[1]{\textcolor[rgb]{0.25,0.44,0.63}{{#1}}}
    \newcommand{\StringTok}[1]{\textcolor[rgb]{0.25,0.44,0.63}{{#1}}}
    \newcommand{\CommentTok}[1]{\textcolor[rgb]{0.38,0.63,0.69}{\textit{{#1}}}}
    \newcommand{\OtherTok}[1]{\textcolor[rgb]{0.00,0.44,0.13}{{#1}}}
    \newcommand{\AlertTok}[1]{\textcolor[rgb]{1.00,0.00,0.00}{\textbf{{#1}}}}
    \newcommand{\FunctionTok}[1]{\textcolor[rgb]{0.02,0.16,0.49}{{#1}}}
    \newcommand{\RegionMarkerTok}[1]{{#1}}
    \newcommand{\ErrorTok}[1]{\textcolor[rgb]{1.00,0.00,0.00}{\textbf{{#1}}}}
    \newcommand{\NormalTok}[1]{{#1}}
    
    % Additional commands for more recent versions of Pandoc
    \newcommand{\ConstantTok}[1]{\textcolor[rgb]{0.53,0.00,0.00}{{#1}}}
    \newcommand{\SpecialCharTok}[1]{\textcolor[rgb]{0.25,0.44,0.63}{{#1}}}
    \newcommand{\VerbatimStringTok}[1]{\textcolor[rgb]{0.25,0.44,0.63}{{#1}}}
    \newcommand{\SpecialStringTok}[1]{\textcolor[rgb]{0.73,0.40,0.53}{{#1}}}
    \newcommand{\ImportTok}[1]{{#1}}
    \newcommand{\DocumentationTok}[1]{\textcolor[rgb]{0.73,0.13,0.13}{\textit{{#1}}}}
    \newcommand{\AnnotationTok}[1]{\textcolor[rgb]{0.38,0.63,0.69}{\textbf{\textit{{#1}}}}}
    \newcommand{\CommentVarTok}[1]{\textcolor[rgb]{0.38,0.63,0.69}{\textbf{\textit{{#1}}}}}
    \newcommand{\VariableTok}[1]{\textcolor[rgb]{0.10,0.09,0.49}{{#1}}}
    \newcommand{\ControlFlowTok}[1]{\textcolor[rgb]{0.00,0.44,0.13}{\textbf{{#1}}}}
    \newcommand{\OperatorTok}[1]{\textcolor[rgb]{0.40,0.40,0.40}{{#1}}}
    \newcommand{\BuiltInTok}[1]{{#1}}
    \newcommand{\ExtensionTok}[1]{{#1}}
    \newcommand{\PreprocessorTok}[1]{\textcolor[rgb]{0.74,0.48,0.00}{{#1}}}
    \newcommand{\AttributeTok}[1]{\textcolor[rgb]{0.49,0.56,0.16}{{#1}}}
    \newcommand{\InformationTok}[1]{\textcolor[rgb]{0.38,0.63,0.69}{\textbf{\textit{{#1}}}}}
    \newcommand{\WarningTok}[1]{\textcolor[rgb]{0.38,0.63,0.69}{\textbf{\textit{{#1}}}}}
    
    
    % Define a nice break command that doesn't care if a line doesn't already
    % exist.
    \def\br{\hspace*{\fill} \\* }
    % Math Jax compatability definitions
    \def\gt{>}
    \def\lt{<}
    % Document parameters
    \title{dataset1\_analyse}
    
    
    

    % Pygments definitions
    
\makeatletter
\def\PY@reset{\let\PY@it=\relax \let\PY@bf=\relax%
    \let\PY@ul=\relax \let\PY@tc=\relax%
    \let\PY@bc=\relax \let\PY@ff=\relax}
\def\PY@tok#1{\csname PY@tok@#1\endcsname}
\def\PY@toks#1+{\ifx\relax#1\empty\else%
    \PY@tok{#1}\expandafter\PY@toks\fi}
\def\PY@do#1{\PY@bc{\PY@tc{\PY@ul{%
    \PY@it{\PY@bf{\PY@ff{#1}}}}}}}
\def\PY#1#2{\PY@reset\PY@toks#1+\relax+\PY@do{#2}}

\expandafter\def\csname PY@tok@w\endcsname{\def\PY@tc##1{\textcolor[rgb]{0.73,0.73,0.73}{##1}}}
\expandafter\def\csname PY@tok@c\endcsname{\let\PY@it=\textit\def\PY@tc##1{\textcolor[rgb]{0.25,0.50,0.50}{##1}}}
\expandafter\def\csname PY@tok@cp\endcsname{\def\PY@tc##1{\textcolor[rgb]{0.74,0.48,0.00}{##1}}}
\expandafter\def\csname PY@tok@k\endcsname{\let\PY@bf=\textbf\def\PY@tc##1{\textcolor[rgb]{0.00,0.50,0.00}{##1}}}
\expandafter\def\csname PY@tok@kp\endcsname{\def\PY@tc##1{\textcolor[rgb]{0.00,0.50,0.00}{##1}}}
\expandafter\def\csname PY@tok@kt\endcsname{\def\PY@tc##1{\textcolor[rgb]{0.69,0.00,0.25}{##1}}}
\expandafter\def\csname PY@tok@o\endcsname{\def\PY@tc##1{\textcolor[rgb]{0.40,0.40,0.40}{##1}}}
\expandafter\def\csname PY@tok@ow\endcsname{\let\PY@bf=\textbf\def\PY@tc##1{\textcolor[rgb]{0.67,0.13,1.00}{##1}}}
\expandafter\def\csname PY@tok@nb\endcsname{\def\PY@tc##1{\textcolor[rgb]{0.00,0.50,0.00}{##1}}}
\expandafter\def\csname PY@tok@nf\endcsname{\def\PY@tc##1{\textcolor[rgb]{0.00,0.00,1.00}{##1}}}
\expandafter\def\csname PY@tok@nc\endcsname{\let\PY@bf=\textbf\def\PY@tc##1{\textcolor[rgb]{0.00,0.00,1.00}{##1}}}
\expandafter\def\csname PY@tok@nn\endcsname{\let\PY@bf=\textbf\def\PY@tc##1{\textcolor[rgb]{0.00,0.00,1.00}{##1}}}
\expandafter\def\csname PY@tok@ne\endcsname{\let\PY@bf=\textbf\def\PY@tc##1{\textcolor[rgb]{0.82,0.25,0.23}{##1}}}
\expandafter\def\csname PY@tok@nv\endcsname{\def\PY@tc##1{\textcolor[rgb]{0.10,0.09,0.49}{##1}}}
\expandafter\def\csname PY@tok@no\endcsname{\def\PY@tc##1{\textcolor[rgb]{0.53,0.00,0.00}{##1}}}
\expandafter\def\csname PY@tok@nl\endcsname{\def\PY@tc##1{\textcolor[rgb]{0.63,0.63,0.00}{##1}}}
\expandafter\def\csname PY@tok@ni\endcsname{\let\PY@bf=\textbf\def\PY@tc##1{\textcolor[rgb]{0.60,0.60,0.60}{##1}}}
\expandafter\def\csname PY@tok@na\endcsname{\def\PY@tc##1{\textcolor[rgb]{0.49,0.56,0.16}{##1}}}
\expandafter\def\csname PY@tok@nt\endcsname{\let\PY@bf=\textbf\def\PY@tc##1{\textcolor[rgb]{0.00,0.50,0.00}{##1}}}
\expandafter\def\csname PY@tok@nd\endcsname{\def\PY@tc##1{\textcolor[rgb]{0.67,0.13,1.00}{##1}}}
\expandafter\def\csname PY@tok@s\endcsname{\def\PY@tc##1{\textcolor[rgb]{0.73,0.13,0.13}{##1}}}
\expandafter\def\csname PY@tok@sd\endcsname{\let\PY@it=\textit\def\PY@tc##1{\textcolor[rgb]{0.73,0.13,0.13}{##1}}}
\expandafter\def\csname PY@tok@si\endcsname{\let\PY@bf=\textbf\def\PY@tc##1{\textcolor[rgb]{0.73,0.40,0.53}{##1}}}
\expandafter\def\csname PY@tok@se\endcsname{\let\PY@bf=\textbf\def\PY@tc##1{\textcolor[rgb]{0.73,0.40,0.13}{##1}}}
\expandafter\def\csname PY@tok@sr\endcsname{\def\PY@tc##1{\textcolor[rgb]{0.73,0.40,0.53}{##1}}}
\expandafter\def\csname PY@tok@ss\endcsname{\def\PY@tc##1{\textcolor[rgb]{0.10,0.09,0.49}{##1}}}
\expandafter\def\csname PY@tok@sx\endcsname{\def\PY@tc##1{\textcolor[rgb]{0.00,0.50,0.00}{##1}}}
\expandafter\def\csname PY@tok@m\endcsname{\def\PY@tc##1{\textcolor[rgb]{0.40,0.40,0.40}{##1}}}
\expandafter\def\csname PY@tok@gh\endcsname{\let\PY@bf=\textbf\def\PY@tc##1{\textcolor[rgb]{0.00,0.00,0.50}{##1}}}
\expandafter\def\csname PY@tok@gu\endcsname{\let\PY@bf=\textbf\def\PY@tc##1{\textcolor[rgb]{0.50,0.00,0.50}{##1}}}
\expandafter\def\csname PY@tok@gd\endcsname{\def\PY@tc##1{\textcolor[rgb]{0.63,0.00,0.00}{##1}}}
\expandafter\def\csname PY@tok@gi\endcsname{\def\PY@tc##1{\textcolor[rgb]{0.00,0.63,0.00}{##1}}}
\expandafter\def\csname PY@tok@gr\endcsname{\def\PY@tc##1{\textcolor[rgb]{1.00,0.00,0.00}{##1}}}
\expandafter\def\csname PY@tok@ge\endcsname{\let\PY@it=\textit}
\expandafter\def\csname PY@tok@gs\endcsname{\let\PY@bf=\textbf}
\expandafter\def\csname PY@tok@gp\endcsname{\let\PY@bf=\textbf\def\PY@tc##1{\textcolor[rgb]{0.00,0.00,0.50}{##1}}}
\expandafter\def\csname PY@tok@go\endcsname{\def\PY@tc##1{\textcolor[rgb]{0.53,0.53,0.53}{##1}}}
\expandafter\def\csname PY@tok@gt\endcsname{\def\PY@tc##1{\textcolor[rgb]{0.00,0.27,0.87}{##1}}}
\expandafter\def\csname PY@tok@err\endcsname{\def\PY@bc##1{\setlength{\fboxsep}{0pt}\fcolorbox[rgb]{1.00,0.00,0.00}{1,1,1}{\strut ##1}}}
\expandafter\def\csname PY@tok@kc\endcsname{\let\PY@bf=\textbf\def\PY@tc##1{\textcolor[rgb]{0.00,0.50,0.00}{##1}}}
\expandafter\def\csname PY@tok@kd\endcsname{\let\PY@bf=\textbf\def\PY@tc##1{\textcolor[rgb]{0.00,0.50,0.00}{##1}}}
\expandafter\def\csname PY@tok@kn\endcsname{\let\PY@bf=\textbf\def\PY@tc##1{\textcolor[rgb]{0.00,0.50,0.00}{##1}}}
\expandafter\def\csname PY@tok@kr\endcsname{\let\PY@bf=\textbf\def\PY@tc##1{\textcolor[rgb]{0.00,0.50,0.00}{##1}}}
\expandafter\def\csname PY@tok@bp\endcsname{\def\PY@tc##1{\textcolor[rgb]{0.00,0.50,0.00}{##1}}}
\expandafter\def\csname PY@tok@fm\endcsname{\def\PY@tc##1{\textcolor[rgb]{0.00,0.00,1.00}{##1}}}
\expandafter\def\csname PY@tok@vc\endcsname{\def\PY@tc##1{\textcolor[rgb]{0.10,0.09,0.49}{##1}}}
\expandafter\def\csname PY@tok@vg\endcsname{\def\PY@tc##1{\textcolor[rgb]{0.10,0.09,0.49}{##1}}}
\expandafter\def\csname PY@tok@vi\endcsname{\def\PY@tc##1{\textcolor[rgb]{0.10,0.09,0.49}{##1}}}
\expandafter\def\csname PY@tok@vm\endcsname{\def\PY@tc##1{\textcolor[rgb]{0.10,0.09,0.49}{##1}}}
\expandafter\def\csname PY@tok@sa\endcsname{\def\PY@tc##1{\textcolor[rgb]{0.73,0.13,0.13}{##1}}}
\expandafter\def\csname PY@tok@sb\endcsname{\def\PY@tc##1{\textcolor[rgb]{0.73,0.13,0.13}{##1}}}
\expandafter\def\csname PY@tok@sc\endcsname{\def\PY@tc##1{\textcolor[rgb]{0.73,0.13,0.13}{##1}}}
\expandafter\def\csname PY@tok@dl\endcsname{\def\PY@tc##1{\textcolor[rgb]{0.73,0.13,0.13}{##1}}}
\expandafter\def\csname PY@tok@s2\endcsname{\def\PY@tc##1{\textcolor[rgb]{0.73,0.13,0.13}{##1}}}
\expandafter\def\csname PY@tok@sh\endcsname{\def\PY@tc##1{\textcolor[rgb]{0.73,0.13,0.13}{##1}}}
\expandafter\def\csname PY@tok@s1\endcsname{\def\PY@tc##1{\textcolor[rgb]{0.73,0.13,0.13}{##1}}}
\expandafter\def\csname PY@tok@mb\endcsname{\def\PY@tc##1{\textcolor[rgb]{0.40,0.40,0.40}{##1}}}
\expandafter\def\csname PY@tok@mf\endcsname{\def\PY@tc##1{\textcolor[rgb]{0.40,0.40,0.40}{##1}}}
\expandafter\def\csname PY@tok@mh\endcsname{\def\PY@tc##1{\textcolor[rgb]{0.40,0.40,0.40}{##1}}}
\expandafter\def\csname PY@tok@mi\endcsname{\def\PY@tc##1{\textcolor[rgb]{0.40,0.40,0.40}{##1}}}
\expandafter\def\csname PY@tok@il\endcsname{\def\PY@tc##1{\textcolor[rgb]{0.40,0.40,0.40}{##1}}}
\expandafter\def\csname PY@tok@mo\endcsname{\def\PY@tc##1{\textcolor[rgb]{0.40,0.40,0.40}{##1}}}
\expandafter\def\csname PY@tok@ch\endcsname{\let\PY@it=\textit\def\PY@tc##1{\textcolor[rgb]{0.25,0.50,0.50}{##1}}}
\expandafter\def\csname PY@tok@cm\endcsname{\let\PY@it=\textit\def\PY@tc##1{\textcolor[rgb]{0.25,0.50,0.50}{##1}}}
\expandafter\def\csname PY@tok@cpf\endcsname{\let\PY@it=\textit\def\PY@tc##1{\textcolor[rgb]{0.25,0.50,0.50}{##1}}}
\expandafter\def\csname PY@tok@c1\endcsname{\let\PY@it=\textit\def\PY@tc##1{\textcolor[rgb]{0.25,0.50,0.50}{##1}}}
\expandafter\def\csname PY@tok@cs\endcsname{\let\PY@it=\textit\def\PY@tc##1{\textcolor[rgb]{0.25,0.50,0.50}{##1}}}

\def\PYZbs{\char`\\}
\def\PYZus{\char`\_}
\def\PYZob{\char`\{}
\def\PYZcb{\char`\}}
\def\PYZca{\char`\^}
\def\PYZam{\char`\&}
\def\PYZlt{\char`\<}
\def\PYZgt{\char`\>}
\def\PYZsh{\char`\#}
\def\PYZpc{\char`\%}
\def\PYZdl{\char`\$}
\def\PYZhy{\char`\-}
\def\PYZsq{\char`\'}
\def\PYZdq{\char`\"}
\def\PYZti{\char`\~}
% for compatibility with earlier versions
\def\PYZat{@}
\def\PYZlb{[}
\def\PYZrb{]}
\makeatother


    % Exact colors from NB
    \definecolor{incolor}{rgb}{0.0, 0.0, 0.5}
    \definecolor{outcolor}{rgb}{0.545, 0.0, 0.0}



    
    % Prevent overflowing lines due to hard-to-break entities
    \sloppy 
    % Setup hyperref package
    \hypersetup{
      breaklinks=true,  % so long urls are correctly broken across lines
      colorlinks=true,
      urlcolor=urlcolor,
      linkcolor=linkcolor,
      citecolor=citecolor,
      }
    % Slightly bigger margins than the latex defaults
    
    \geometry{verbose,tmargin=1in,bmargin=1in,lmargin=1in,rmargin=1in}
    
    

    \begin{document}
    
    
    \maketitle
    
    

    
    \section{数据挖掘作业 1
数据探索性分析与预处理}\label{ux6570ux636eux6316ux6398ux4f5cux4e1a-1-ux6570ux636eux63a2ux7d22ux6027ux5206ux6790ux4e0eux9884ux5904ux7406}

\textbf{姓名:刘张敏}

\textbf{学号:3220180832}

\textbf{日期:2019.3.29}

    \subsubsection{数据分析要求}\label{ux6570ux636eux5206ux6790ux8981ux6c42}

\textbf{1. 数据可视化和摘要}

数据摘要

\begin{itemize}
\item
  对标称属性,给出每个可能取值的频数
\item
  对数值属性,给出最大、最小、均值、中位数、四分位数及缺失值的个数。
\end{itemize}

数据的可视化

针对数值属性:

\begin{itemize}
\item
  绘制直方图,用qq图检验其分布是否为正态分布。
\item
  绘制盒图,对离群值进行识别。
\end{itemize}

\textbf{2. 数据缺失的处理}

观察数据集中缺失数据,分析其缺失的原因。

分别使用下列四种策略对缺失值进行处理

\begin{itemize}
\tightlist
\item
  将缺失部分剔除
\item
  用最高频率值来填补缺失值
\item
  通过属性的相关关系来填补缺失值
\item
  通过数据对象之间的相似性来填补缺失值
\end{itemize}

    \subsubsection{解答内容}\label{ux89e3ux7b54ux5185ux5bb9}

    \begin{Verbatim}[commandchars=\\\{\}]
{\color{incolor}In [{\color{incolor}4}]:} \PY{c+ch}{\PYZsh{}!/usr/bin/env python}
        \PY{c+c1}{\PYZsh{} \PYZhy{}*\PYZhy{} coding:utf\PYZhy{}8 \PYZhy{}*\PYZhy{}}
        \PY{k+kn}{import} \PY{n+nn}{operator}
        \PY{k+kn}{import} \PY{n+nn}{pandas} \PY{k}{as} \PY{n+nn}{pd}
        \PY{k+kn}{import} \PY{n+nn}{numpy} \PY{k}{as} \PY{n+nn}{np}
        \PY{k+kn}{import} \PY{n+nn}{statsmodels}\PY{n+nn}{.}\PY{n+nn}{api} \PY{k}{as} \PY{n+nn}{sm}
        \PY{k+kn}{import} \PY{n+nn}{scipy}\PY{n+nn}{.}\PY{n+nn}{stats} \PY{k}{as} \PY{n+nn}{stats}
        \PY{k+kn}{import} \PY{n+nn}{matplotlib}\PY{n+nn}{.}\PY{n+nn}{pyplot} \PY{k}{as} \PY{n+nn}{plt}
        \PY{k+kn}{import} \PY{n+nn}{matplotlib}
        \PY{n}{matplotlib}\PY{o}{.}\PY{n}{style}\PY{o}{.}\PY{n}{use}\PY{p}{(}\PY{l+s+s1}{\PYZsq{}}\PY{l+s+s1}{ggplot}\PY{l+s+s1}{\PYZsq{}}\PY{p}{)}
        \PY{o}{\PYZpc{}}\PY{k}{pylab} inline
\end{Verbatim}


    \begin{Verbatim}[commandchars=\\\{\}]
Populating the interactive namespace from numpy and matplotlib

    \end{Verbatim}

    \textbf{Step1. 读取数据}

\begin{itemize}
\tightlist
\item
  读取csv文件,生成data frame
\end{itemize}

    \begin{Verbatim}[commandchars=\\\{\}]
{\color{incolor}In [{\color{incolor}28}]:} \PY{c+c1}{\PYZsh{} 定义两类数据:标称型和数值型}
         \PY{n}{name\PYZus{}category} \PY{o}{=} \PY{p}{[}\PY{l+s+s2}{\PYZdq{}}\PY{l+s+s2}{Agency}\PY{l+s+s2}{\PYZdq{}}\PY{p}{,} \PY{l+s+s2}{\PYZdq{}}\PY{l+s+s2}{Location}\PY{l+s+s2}{\PYZdq{}}\PY{p}{,} \PY{l+s+s2}{\PYZdq{}}\PY{l+s+s2}{Area Id}\PY{l+s+s2}{\PYZdq{}}\PY{p}{,} \PY{l+s+s2}{\PYZdq{}}\PY{l+s+s2}{Beat}\PY{l+s+s2}{\PYZdq{}}\PY{p}{,} \PY{l+s+s2}{\PYZdq{}}\PY{l+s+s2}{Incident Type Id}\PY{l+s+s2}{\PYZdq{}}\PY{p}{,} \PY{l+s+s2}{\PYZdq{}}\PY{l+s+s2}{Event Number}\PY{l+s+s2}{\PYZdq{}}\PY{p}{]}
         \PY{n}{name\PYZus{}value} \PY{o}{=} \PY{p}{[}\PY{l+s+s2}{\PYZdq{}}\PY{l+s+s2}{Priority}\PY{l+s+s2}{\PYZdq{}}\PY{p}{]}
         
         \PY{c+c1}{\PYZsh{} 读取数据}
         \PY{n}{data\PYZus{}origin} \PY{o}{=} \PY{n}{pd}\PY{o}{.}\PY{n}{read\PYZus{}csv}\PY{p}{(}\PY{l+s+s2}{\PYZdq{}}\PY{l+s+s2}{./data\PYZus{}origin/records\PYZhy{}for\PYZhy{}2016.csv}\PY{l+s+s2}{\PYZdq{}}\PY{p}{,}
                            \PY{n}{na\PYZus{}values} \PY{o}{=} \PY{l+s+s2}{\PYZdq{}}\PY{l+s+s2}{XXXXXXX}\PY{l+s+s2}{\PYZdq{}}\PY{p}{)}
         
         \PY{c+c1}{\PYZsh{} 将字符数据转换为category}
         \PY{k}{for} \PY{n}{item} \PY{o+ow}{in} \PY{n}{name\PYZus{}category}\PY{p}{:}
             \PY{n}{data\PYZus{}origin}\PY{p}{[}\PY{n}{item}\PY{p}{]} \PY{o}{=} \PY{n}{data\PYZus{}origin}\PY{p}{[}\PY{n}{item}\PY{p}{]}\PY{o}{.}\PY{n}{astype}\PY{p}{(}\PY{l+s+s1}{\PYZsq{}}\PY{l+s+s1}{category}\PY{l+s+s1}{\PYZsq{}}\PY{p}{)}
         
         \PY{c+c1}{\PYZsh{} 查看前10条数据内容}
         \PY{n}{data\PYZus{}origin}\PY{o}{.}\PY{n}{iloc}\PY{p}{[}\PY{p}{:}\PY{l+m+mi}{10}\PY{p}{]}
\end{Verbatim}


\begin{Verbatim}[commandchars=\\\{\}]
{\color{outcolor}Out[{\color{outcolor}28}]:}   Agency              Create Time             Location Area Id Beat  Priority  \textbackslash{}
         0     OP  2016-01-01T00:00:57.000     ST\&MARKET ST          P1  05X       2.0   
         1     OP  2016-01-01T00:01:25.000   AV\&HAMILTON ST          P3  26Y       2.0   
         2     OP  2016-01-01T00:01:43.000   ST\&CHESTNUT ST          P1  02X       2.0   
         3     OP  2016-01-01T00:01:48.000       WALLACE ST          P2  18Y       2.0   
         4     OP  2016-01-01T00:02:05.000          90TH AV          P3  34X       2.0   
         5     OP  2016-01-01T00:02:55.000        PARK BLVD          P2  16Y       2.0   
         6     OP  2016-01-01T00:03:20.000        PIPPIN ST          P3  31Z       2.0   
         7     OP  2016-01-01T00:03:32.000        POTTER ST          P3  27X       2.0   
         8     OP  2016-01-01T00:04:35.000       OUTLOOK AV          P3  30Y       2.0   
         9     OP  2016-01-01T00:04:46.000          61ST ST          P1  10X       2.0   
         
           Incident Type Id Incident Type Description     Event Number  \textbackslash{}
         0            415GS              415 GUNSHOTS  LOP160101000003   
         1            415GS              415 GUNSHOTS  LOP160101000005   
         2            415GS              415 GUNSHOTS  LOP160101000008   
         3            415GS              415 GUNSHOTS  LOP160101000007   
         4            415GS              415 GUNSHOTS  LOP160101000009   
         5              314         INDECENT EXPOSURE  LOP160101000010   
         6            415GS              415 GUNSHOTS  LOP160101000011   
         7            415FC         415 FIRE CRACKERS  LOP160101000012   
         8             933R              ALARM-RINGER  LOP160101000014   
         9            415FC         415 FIRE CRACKERS  LOP160101000017   
         
                        Closed Time  
         0  2016-01-01T00:32:30.000  
         1  2016-01-01T00:48:23.000  
         2  2016-01-01T00:21:24.000  
         3  2016-01-01T01:15:03.000  
         4  2016-01-01T00:54:52.000  
         5  2016-01-01T01:53:59.000  
         6  2016-01-01T01:11:59.000  
         7  2016-01-01T01:43:12.000  
         8  2016-01-01T00:11:13.000  
         9  2016-01-01T00:53:39.000  
\end{Verbatim}
            
    \textbf{Step 2. 数据摘要}

\begin{itemize}
\tightlist
\item
  对标称属性,给出每个可能取值的频数
\end{itemize}

    \begin{Verbatim}[commandchars=\\\{\}]
{\color{incolor}In [{\color{incolor}32}]:} \PY{c+c1}{\PYZsh{} 使用value\PYZus{}counts函数统计每个标称属性的取值频数}
         \PY{k}{for} \PY{n}{item} \PY{o+ow}{in} \PY{n}{name\PYZus{}category}\PY{p}{:}
             \PY{n+nb}{print} \PY{p}{(}\PY{n}{item} \PY{o}{+} \PY{l+s+s2}{\PYZdq{}}\PY{l+s+s2}{的频数为:}\PY{l+s+se}{\PYZbs{}n}\PY{l+s+s2}{\PYZdq{}}\PY{p}{)} 
             \PY{n+nb}{print} \PY{p}{(}\PY{n}{pd}\PY{o}{.}\PY{n}{value\PYZus{}counts}\PY{p}{(}\PY{n}{data\PYZus{}origin}\PY{p}{[}\PY{n}{item}\PY{p}{]}\PY{o}{.}\PY{n}{values}\PY{p}{)}\PY{p}{)}
\end{Verbatim}


    \begin{Verbatim}[commandchars=\\\{\}]
Agency的频数为:

OP    110827
dtype: int64
Location的频数为:

 INTERNATIONAL BLVD                  2156
 AV\&INTERNATIONAL BLVD               1829
 MACARTHUR BLVD                      1813
 BROADWAY                            1472
 7TH ST                              1223
 FOOTHILL BLVD                       1052
 TELEGRAPH AV                         875
 SAN PABLO AV                         765
 AV\&MACARTHUR BLVD                    737
 FRUITVALE AV                         709
 BANCROFT AV                          707
 ST\&BROADWAY                          691
 HIGH ST                              678
 ST\&TELEGRAPH AV                      638
 73RD AV                              595
 HEGENBERGER RD                       591
 AV\&FOOTHILL BLVD                     554
 LAKESHORE AV                         509
 WEBSTER ST                           465
 E 12TH ST                            455
 AV\&BANCROFT AV                       440
 HARRISON ST                          436
 ST\&MARKET ST                         427
 GRAND AV                             416
 MARKET ST                            414
 35TH AV                              408
 14TH ST                              401
 ST\&MARTIN LUTHER KING JR WY          401
 34TH ST                              371
 ST\&SAN PABLO AV                      363
                                     {\ldots} 
94TH 12TH AV                            1
94TH 13TH AV                            1
94TH 16TH AV                            1
94TH 23RD ST\&INYO AV                    1
94TH 28TH AV                            1
94TH 32ND ST                            1
94TH 33RD AV                            1
94TH 33RD ST\&PARK BLVD                  1
93RD E 17TH ST                          1
93RD AV\&SAN LEANDRO ST                  1
93RD AV\&CHURCH ST                       1
93RD AV\&BANCROFT AV                     1
92ND ST\&LINCOLN AV                      1
92ND ST\&MADISON ST                      1
92ND ST\&MARKET ST                       1
92ND ST\&MARTIN LUTHER KING JR WY        1
92ND ST\&VAN BUREN AV                    1
92ND SYCAMORE ST                        1
92ND TELEGRAPH AV                       1
92ND TERRACE                            1
92ND VICTOR AV                          1
92ND W GRAND AV                         1
92ND WAKEFIELD AV                       1
92ND WALNUT ST                          1
934A MARTIN LUTHER KING JR WY           1
93RD 105TH AV                           1
93RD 40TH STREET WY                     1
93RD 41ST AV                            1
93RD 66TH AV                            1
62ND BARTLETT ST                        1
Length: 24046, dtype: int64
Area Id的频数为:

P3     47425
P1     41419
P2     19610
POU     2173
PCW      194
TEC        4
WAG        1
JLS        1
dtype: int64
Beat的频数为:

04X     4515
08X     3931
26Y     3511
30Y     3473
19X     3455
30X     3416
03X     3195
23X     3076
34X     2857
07X     2831
20X     2702
29X     2646
06X     2580
03Y     2562
27Y     2517
25X     2467
31Y     2460
27X     2333
35X     2328
32X     2316
33X     2276
09X     2158
21Y     2100
32Y     2093
12Y     1987
14X     1832
26X     1766
02X     1746
24X     1704
02Y     1659
10Y     1573
10X     1557
22X     1541
17Y     1482
21X     1479
24Y     1454
31X     1439
22Y     1420
13Z     1397
15X     1393
05X     1342
01X     1304
12X     1299
31Z     1268
28X     1261
11X     1208
35Y     1159
18Y     1102
14Y     1027
17X      969
13Y      952
16Y      907
25Y      739
18X      721
16X      708
13X      630
05Y      408
PDT2      16
dtype: int64
Incident Type Id的频数为:

933R      10094
415        7883
SECCK      7251
10851      5308
911H       5089
5150       4859
415C       3701
242        3483
912        2583
949        2504
HAZ        2424
243E       2310
WELCK      2276
415GS      2025
901A       1730
602L       1690
211        1592
415TH      1458
943        1288
415N       1287
245        1286
975        1159
415CU      1084
20002      1071
933MA      1061
601R       1024
901        1020
EVAL       1011
CODE7       975
597         810
          {\ldots}  
11351         3
148\_5A        3
962           3
182           3
243           3
21235V        3
529           3
203           3
10852         2
140           2
10801         2
WIT           2
SUSPS         2
484G          2
503           2
591           2
626\_6         2
A487          1
OTC           1
EBMUD         1
DROWN         1
ABC           1
243A          1
970A          1
955B          1
300WI         1
487E          1
407           1
3211H         1
YELALT        1
Length: 242, dtype: int64
Event Number的频数为:

LOP160731000897    1
LOP160315000161    1
LOP160315000126    1
LOP160315000130    1
LOP160315000135    1
LOP160315000139    1
LOP160315000140    1
LOP160315000147    1
LOP160315000152    1
LOP160315000154    1
LOP160315000156    1
LOP160315000157    1
LOP160315000158    1
LOP160315000166    1
LOP160314001004    1
LOP160315000167    1
LOP160315000169    1
LOP160315000172    1
LOP160315000174    1
LOP160315000177    1
LOP160315000178    1
LOP160315000181    1
LOP160315000182    1
LOP160315000184    1
LOP160315000186    1
LOP160315000187    1
LOP160315000125    1
LOP160315000124    1
LOP160315000123    1
LOP160315000122    1
                  ..
LOP160523000540    1
LOP160523000541    1
LOP160523000480    1
LOP160523000477    1
LOP160523000428    1
LOP160523000476    1
LOP160523000429    1
LOP160523000430    1
LOP160523000432    1
LOP160523000434    1
LOP160523000437    1
LOP160523000439    1
LOP160523000441    1
LOP160523000443    1
LOP160523000445    1
LOP160523000447    1
LOP160523000448    1
LOP160523000449    1
LOP160523000451    1
LOP160523000452    1
LOP160523000458    1
LOP160523000459    1
LOP160523000464    1
LOP160523000465    1
LOP160523000467    1
LOP160523000469    1
LOP160523000470    1
LOP160523000472    1
LOP160523000473    1
LOP160101000003    1
Length: 110827, dtype: int64

    \end{Verbatim}

    \begin{itemize}
\tightlist
\item
  对数值属性,给出最大、最小、均值、中位数、四分位数及缺失值的个数。
\end{itemize}

    \begin{Verbatim}[commandchars=\\\{\}]
{\color{incolor}In [{\color{incolor}33}]:} \PY{c+c1}{\PYZsh{} 最大值}
         \PY{n}{data\PYZus{}show} \PY{o}{=} \PY{n}{pd}\PY{o}{.}\PY{n}{DataFrame}\PY{p}{(}\PY{n}{data} \PY{o}{=} \PY{n}{data\PYZus{}origin}\PY{p}{[}\PY{n}{name\PYZus{}value}\PY{p}{]}\PY{o}{.}\PY{n}{max}\PY{p}{(}\PY{p}{)}\PY{p}{,} \PY{n}{columns} \PY{o}{=} \PY{p}{[}\PY{l+s+s1}{\PYZsq{}}\PY{l+s+s1}{max}\PY{l+s+s1}{\PYZsq{}}\PY{p}{]}\PY{p}{)}
         \PY{c+c1}{\PYZsh{} 最小值}
         \PY{n}{data\PYZus{}show}\PY{p}{[}\PY{l+s+s1}{\PYZsq{}}\PY{l+s+s1}{min}\PY{l+s+s1}{\PYZsq{}}\PY{p}{]} \PY{o}{=} \PY{n}{data\PYZus{}origin}\PY{p}{[}\PY{n}{name\PYZus{}value}\PY{p}{]}\PY{o}{.}\PY{n}{min}\PY{p}{(}\PY{p}{)}
         \PY{c+c1}{\PYZsh{} 均值}
         \PY{n}{data\PYZus{}show}\PY{p}{[}\PY{l+s+s1}{\PYZsq{}}\PY{l+s+s1}{mean}\PY{l+s+s1}{\PYZsq{}}\PY{p}{]} \PY{o}{=} \PY{n}{data\PYZus{}origin}\PY{p}{[}\PY{n}{name\PYZus{}value}\PY{p}{]}\PY{o}{.}\PY{n}{mean}\PY{p}{(}\PY{p}{)}
         \PY{c+c1}{\PYZsh{} 中位数}
         \PY{n}{data\PYZus{}show}\PY{p}{[}\PY{l+s+s1}{\PYZsq{}}\PY{l+s+s1}{median}\PY{l+s+s1}{\PYZsq{}}\PY{p}{]} \PY{o}{=} \PY{n}{data\PYZus{}origin}\PY{p}{[}\PY{n}{name\PYZus{}value}\PY{p}{]}\PY{o}{.}\PY{n}{median}\PY{p}{(}\PY{p}{)}
         \PY{c+c1}{\PYZsh{} 四分位数}
         \PY{n}{data\PYZus{}show}\PY{p}{[}\PY{l+s+s1}{\PYZsq{}}\PY{l+s+s1}{quartile}\PY{l+s+s1}{\PYZsq{}}\PY{p}{]} \PY{o}{=} \PY{n}{data\PYZus{}origin}\PY{p}{[}\PY{n}{name\PYZus{}value}\PY{p}{]}\PY{o}{.}\PY{n}{describe}\PY{p}{(}\PY{p}{)}\PY{o}{.}\PY{n}{loc}\PY{p}{[}\PY{l+s+s1}{\PYZsq{}}\PY{l+s+s1}{25}\PY{l+s+s1}{\PYZpc{}}\PY{l+s+s1}{\PYZsq{}}\PY{p}{]}
         \PY{c+c1}{\PYZsh{} 缺失值个数}
         \PY{n}{data\PYZus{}show}\PY{p}{[}\PY{l+s+s1}{\PYZsq{}}\PY{l+s+s1}{missing}\PY{l+s+s1}{\PYZsq{}}\PY{p}{]} \PY{o}{=} \PY{n}{data\PYZus{}origin}\PY{p}{[}\PY{n}{name\PYZus{}value}\PY{p}{]}\PY{o}{.}\PY{n}{describe}\PY{p}{(}\PY{p}{)}\PY{o}{.}\PY{n}{loc}\PY{p}{[}\PY{l+s+s1}{\PYZsq{}}\PY{l+s+s1}{count}\PY{l+s+s1}{\PYZsq{}}\PY{p}{]}\PY{o}{.}\PY{n}{apply}\PY{p}{(}\PY{k}{lambda} \PY{n}{x} \PY{p}{:} \PY{l+m+mi}{200}\PY{o}{\PYZhy{}}\PY{n}{x}\PY{p}{)}
\end{Verbatim}


    \begin{Verbatim}[commandchars=\\\{\}]
{\color{incolor}In [{\color{incolor}34}]:} \PY{n}{data\PYZus{}show}
\end{Verbatim}


\begin{Verbatim}[commandchars=\\\{\}]
{\color{outcolor}Out[{\color{outcolor}34}]:}           max  min      mean  median  quartile   missing
         Priority  2.0  1.0  1.778438     2.0       2.0 -110627.0
\end{Verbatim}
            
    \textbf{Step 3. 数据可视化 }

\begin{itemize}
\tightlist
\item
  针对数值属性: 绘制直方图,用qq图检验其分布是否为正态分布。
\end{itemize}

    \begin{Verbatim}[commandchars=\\\{\}]
{\color{incolor}In [{\color{incolor}35}]:} \PY{c+c1}{\PYZsh{} 直方图}
         \PY{n}{fig} \PY{o}{=} \PY{n}{plt}\PY{o}{.}\PY{n}{figure}\PY{p}{(}\PY{n}{figsize} \PY{o}{=} \PY{p}{(}\PY{l+m+mi}{20}\PY{p}{,}\PY{l+m+mi}{11}\PY{p}{)}\PY{p}{)}
         \PY{n}{i} \PY{o}{=} \PY{l+m+mi}{1}
         \PY{k}{for} \PY{n}{item} \PY{o+ow}{in} \PY{n}{name\PYZus{}value}\PY{p}{:}
             \PY{n}{ax} \PY{o}{=} \PY{n}{fig}\PY{o}{.}\PY{n}{add\PYZus{}subplot}\PY{p}{(}\PY{l+m+mi}{3}\PY{p}{,} \PY{l+m+mi}{5}\PY{p}{,} \PY{n}{i}\PY{p}{)}
             \PY{n}{data\PYZus{}origin}\PY{p}{[}\PY{n}{item}\PY{p}{]}\PY{o}{.}\PY{n}{plot}\PY{p}{(}\PY{n}{kind} \PY{o}{=} \PY{l+s+s1}{\PYZsq{}}\PY{l+s+s1}{hist}\PY{l+s+s1}{\PYZsq{}}\PY{p}{,} \PY{n}{title} \PY{o}{=} \PY{n}{item}\PY{p}{,} \PY{n}{ax} \PY{o}{=} \PY{n}{ax}\PY{p}{)}
             \PY{n}{i} \PY{o}{+}\PY{o}{=} \PY{l+m+mi}{1}
         \PY{n}{plt}\PY{o}{.}\PY{n}{subplots\PYZus{}adjust}\PY{p}{(}\PY{n}{wspace} \PY{o}{=} \PY{l+m+mf}{0.3}\PY{p}{,} \PY{n}{hspace} \PY{o}{=} \PY{l+m+mf}{0.3}\PY{p}{)}
         \PY{n}{fig}\PY{o}{.}\PY{n}{savefig}\PY{p}{(}\PY{l+s+s1}{\PYZsq{}}\PY{l+s+s1}{./image/histogram.jpg}\PY{l+s+s1}{\PYZsq{}}\PY{p}{)}
\end{Verbatim}


    \begin{center}
    \adjustimage{max size={0.9\linewidth}{0.9\paperheight}}{output_12_0.png}
    \end{center}
    { \hspace*{\fill} \\}
    
    \begin{Verbatim}[commandchars=\\\{\}]
{\color{incolor}In [{\color{incolor}36}]:} \PY{c+c1}{\PYZsh{} qq图}
         \PY{n}{fig} \PY{o}{=} \PY{n}{plt}\PY{o}{.}\PY{n}{figure}\PY{p}{(}\PY{n}{figsize} \PY{o}{=} \PY{p}{(}\PY{l+m+mi}{20}\PY{p}{,}\PY{l+m+mi}{12}\PY{p}{)}\PY{p}{)}
         \PY{n}{i} \PY{o}{=} \PY{l+m+mi}{1}
         \PY{k}{for} \PY{n}{item} \PY{o+ow}{in} \PY{n}{name\PYZus{}value}\PY{p}{:}
             \PY{n}{ax} \PY{o}{=} \PY{n}{fig}\PY{o}{.}\PY{n}{add\PYZus{}subplot}\PY{p}{(}\PY{l+m+mi}{3}\PY{p}{,} \PY{l+m+mi}{5}\PY{p}{,} \PY{n}{i}\PY{p}{)}
             \PY{n}{sm}\PY{o}{.}\PY{n}{qqplot}\PY{p}{(}\PY{n}{data\PYZus{}origin}\PY{p}{[}\PY{n}{item}\PY{p}{]}\PY{p}{,} \PY{n}{ax} \PY{o}{=} \PY{n}{ax}\PY{p}{)}
             \PY{n}{ax}\PY{o}{.}\PY{n}{set\PYZus{}title}\PY{p}{(}\PY{n}{item}\PY{p}{)}
             \PY{n}{i} \PY{o}{+}\PY{o}{=} \PY{l+m+mi}{1}
         \PY{n}{plt}\PY{o}{.}\PY{n}{subplots\PYZus{}adjust}\PY{p}{(}\PY{n}{wspace} \PY{o}{=} \PY{l+m+mf}{0.3}\PY{p}{,} \PY{n}{hspace} \PY{o}{=} \PY{l+m+mf}{0.3}\PY{p}{)}
         \PY{n}{fig}\PY{o}{.}\PY{n}{savefig}\PY{p}{(}\PY{l+s+s1}{\PYZsq{}}\PY{l+s+s1}{./image/qqplot.jpg}\PY{l+s+s1}{\PYZsq{}}\PY{p}{)}
\end{Verbatim}


    \begin{center}
    \adjustimage{max size={0.9\linewidth}{0.9\paperheight}}{output_13_0.png}
    \end{center}
    { \hspace*{\fill} \\}
    
    从qq图中可以看出,只有mxPH和mnO2两项值符合正态分布,其他值均不符合

    \begin{itemize}
\tightlist
\item
  绘制盒图,对离群值进行识别。
\end{itemize}

    \begin{Verbatim}[commandchars=\\\{\}]
{\color{incolor}In [{\color{incolor}37}]:} \PY{c+c1}{\PYZsh{} 盒图}
         \PY{n}{fig} \PY{o}{=} \PY{n}{plt}\PY{o}{.}\PY{n}{figure}\PY{p}{(}\PY{n}{figsize} \PY{o}{=} \PY{p}{(}\PY{l+m+mi}{20}\PY{p}{,}\PY{l+m+mi}{12}\PY{p}{)}\PY{p}{)}
         \PY{n}{i} \PY{o}{=} \PY{l+m+mi}{1}
         \PY{k}{for} \PY{n}{item} \PY{o+ow}{in} \PY{n}{name\PYZus{}value}\PY{p}{:}
             \PY{n}{ax} \PY{o}{=} \PY{n}{fig}\PY{o}{.}\PY{n}{add\PYZus{}subplot}\PY{p}{(}\PY{l+m+mi}{3}\PY{p}{,} \PY{l+m+mi}{5}\PY{p}{,} \PY{n}{i}\PY{p}{)}
             \PY{n}{data\PYZus{}origin}\PY{p}{[}\PY{n}{item}\PY{p}{]}\PY{o}{.}\PY{n}{plot}\PY{p}{(}\PY{n}{kind} \PY{o}{=} \PY{l+s+s1}{\PYZsq{}}\PY{l+s+s1}{box}\PY{l+s+s1}{\PYZsq{}}\PY{p}{)}
             \PY{n}{i} \PY{o}{+}\PY{o}{=} \PY{l+m+mi}{1}
         \PY{n}{fig}\PY{o}{.}\PY{n}{savefig}\PY{p}{(}\PY{l+s+s1}{\PYZsq{}}\PY{l+s+s1}{./image/boxplot.jpg}\PY{l+s+s1}{\PYZsq{}}\PY{p}{)}
\end{Verbatim}


    \begin{center}
    \adjustimage{max size={0.9\linewidth}{0.9\paperheight}}{output_16_0.png}
    \end{center}
    { \hspace*{\fill} \\}
    
    \textbf{Step 4. 数据缺失的处理}

可视化方法:对于\textbf{标称属性},绘制属性的折线图,图中红线是原始数据,蓝线是处理完缺失值之后的数据;\textbf{数值属性}:使用直方图,将原始数据和处理后的数据图像进行叠加。图中红色的垂线是原始数据的均值,蓝色的垂线是处理完缺失值之后的均值。

    4.0 观察数据

从绘制的表格上可以看出,缺失值主要集中在Beat个属性,第238条数据缺失情况比较严重

    \begin{Verbatim}[commandchars=\\\{\}]
{\color{incolor}In [{\color{incolor}38}]:} \PY{c+c1}{\PYZsh{} 找出含有缺失值的数据条目索引值}
         \PY{n}{nan\PYZus{}list} \PY{o}{=} \PY{n}{pd}\PY{o}{.}\PY{n}{isnull}\PY{p}{(}\PY{n}{data\PYZus{}origin}\PY{p}{)}\PY{o}{.}\PY{n}{any}\PY{p}{(}\PY{l+m+mi}{1}\PY{p}{)}\PY{o}{.}\PY{n}{nonzero}\PY{p}{(}\PY{p}{)}\PY{p}{[}\PY{l+m+mi}{0}\PY{p}{]}
         
         \PY{c+c1}{\PYZsh{} 显示含有缺失值的原始数据条目}
         \PY{n}{data\PYZus{}origin}\PY{o}{.}\PY{n}{iloc}\PY{p}{[}\PY{n}{nan\PYZus{}list}\PY{p}{]}\PY{o}{.}\PY{n}{style}\PY{o}{.}\PY{n}{highlight\PYZus{}null}\PY{p}{(}\PY{n}{null\PYZus{}color}\PY{o}{=}\PY{l+s+s1}{\PYZsq{}}\PY{l+s+s1}{red}\PY{l+s+s1}{\PYZsq{}}\PY{p}{)}
\end{Verbatim}


\begin{Verbatim}[commandchars=\\\{\}]
{\color{outcolor}Out[{\color{outcolor}38}]:} <pandas.io.formats.style.Styler at 0x1adf104a7f0>
\end{Verbatim}
            
    4.1 将缺失部分剔除

使用\textbf{\emph{dropna()}}函数操作。从结果可以看出,由于删除了带有缺失值的整条数据。

从标称属性的折线图,可以明显看出处理后的数据量减少;直方图中,蓝色线和红色线不重合,但是十分接近,说明数值属性的均值有改变,但是变化不大。

    \begin{Verbatim}[commandchars=\\\{\}]
{\color{incolor}In [{\color{incolor}39}]:} \PY{c+c1}{\PYZsh{} 将缺失值对应的数据整条剔除,生成新数据集}
         \PY{n}{data\PYZus{}filtrated} \PY{o}{=} \PY{n}{data\PYZus{}origin}\PY{o}{.}\PY{n}{dropna}\PY{p}{(}\PY{p}{)}
         
         \PY{c+c1}{\PYZsh{} 绘制可视化图}
         \PY{n}{fig} \PY{o}{=} \PY{n}{plt}\PY{o}{.}\PY{n}{figure}\PY{p}{(}\PY{n}{figsize} \PY{o}{=} \PY{p}{(}\PY{l+m+mi}{20}\PY{p}{,}\PY{l+m+mi}{15}\PY{p}{)}\PY{p}{)}
         
         \PY{n}{i} \PY{o}{=} \PY{l+m+mi}{1}
         \PY{c+c1}{\PYZsh{} 对标称属性,绘制折线图}
         \PY{k}{for} \PY{n}{item} \PY{o+ow}{in} \PY{n}{name\PYZus{}category}\PY{p}{:}
             \PY{n}{ax} \PY{o}{=} \PY{n}{fig}\PY{o}{.}\PY{n}{add\PYZus{}subplot}\PY{p}{(}\PY{l+m+mi}{4}\PY{p}{,} \PY{l+m+mi}{5}\PY{p}{,} \PY{n}{i}\PY{p}{)}
             \PY{n}{ax}\PY{o}{.}\PY{n}{set\PYZus{}title}\PY{p}{(}\PY{n}{item}\PY{p}{)}
             \PY{n}{pd}\PY{o}{.}\PY{n}{value\PYZus{}counts}\PY{p}{(}\PY{n}{data\PYZus{}origin}\PY{p}{[}\PY{n}{item}\PY{p}{]}\PY{o}{.}\PY{n}{values}\PY{p}{)}\PY{o}{.}\PY{n}{plot}\PY{p}{(}\PY{n}{ax} \PY{o}{=} \PY{n}{ax}\PY{p}{,} \PY{n}{marker} \PY{o}{=} \PY{l+s+s1}{\PYZsq{}}\PY{l+s+s1}{\PYZca{}}\PY{l+s+s1}{\PYZsq{}}\PY{p}{,} \PY{n}{label} \PY{o}{=} \PY{l+s+s1}{\PYZsq{}}\PY{l+s+s1}{origin}\PY{l+s+s1}{\PYZsq{}}\PY{p}{,} \PY{n}{legend} \PY{o}{=} \PY{k+kc}{True}\PY{p}{)}
             \PY{n}{pd}\PY{o}{.}\PY{n}{value\PYZus{}counts}\PY{p}{(}\PY{n}{data\PYZus{}filtrated}\PY{p}{[}\PY{n}{item}\PY{p}{]}\PY{o}{.}\PY{n}{values}\PY{p}{)}\PY{o}{.}\PY{n}{plot}\PY{p}{(}\PY{n}{ax} \PY{o}{=} \PY{n}{ax}\PY{p}{,} \PY{n}{marker} \PY{o}{=} \PY{l+s+s1}{\PYZsq{}}\PY{l+s+s1}{o}\PY{l+s+s1}{\PYZsq{}}\PY{p}{,} \PY{n}{label} \PY{o}{=} \PY{l+s+s1}{\PYZsq{}}\PY{l+s+s1}{filtrated}\PY{l+s+s1}{\PYZsq{}}\PY{p}{,} \PY{n}{legend} \PY{o}{=} \PY{k+kc}{True}\PY{p}{)}
             \PY{n}{i} \PY{o}{+}\PY{o}{=} \PY{l+m+mi}{1}
         
         \PY{n}{i} \PY{o}{=} \PY{l+m+mi}{6}
         \PY{c+c1}{\PYZsh{} 对数值属性,绘制直方图}
         \PY{k}{for} \PY{n}{item} \PY{o+ow}{in} \PY{n}{name\PYZus{}value}\PY{p}{:}
             \PY{n}{ax} \PY{o}{=} \PY{n}{fig}\PY{o}{.}\PY{n}{add\PYZus{}subplot}\PY{p}{(}\PY{l+m+mi}{4}\PY{p}{,} \PY{l+m+mi}{5}\PY{p}{,} \PY{n}{i}\PY{p}{)}
             \PY{n}{ax}\PY{o}{.}\PY{n}{set\PYZus{}title}\PY{p}{(}\PY{n}{item}\PY{p}{)}
             \PY{n}{data\PYZus{}origin}\PY{p}{[}\PY{n}{item}\PY{p}{]}\PY{o}{.}\PY{n}{plot}\PY{p}{(}\PY{n}{ax} \PY{o}{=} \PY{n}{ax}\PY{p}{,} \PY{n}{alpha} \PY{o}{=} \PY{l+m+mf}{0.5}\PY{p}{,} \PY{n}{kind} \PY{o}{=} \PY{l+s+s1}{\PYZsq{}}\PY{l+s+s1}{hist}\PY{l+s+s1}{\PYZsq{}}\PY{p}{,} \PY{n}{label} \PY{o}{=} \PY{l+s+s1}{\PYZsq{}}\PY{l+s+s1}{origin}\PY{l+s+s1}{\PYZsq{}}\PY{p}{,} \PY{n}{legend} \PY{o}{=} \PY{k+kc}{True}\PY{p}{)}
             \PY{n}{data\PYZus{}filtrated}\PY{p}{[}\PY{n}{item}\PY{p}{]}\PY{o}{.}\PY{n}{plot}\PY{p}{(}\PY{n}{ax} \PY{o}{=} \PY{n}{ax}\PY{p}{,} \PY{n}{alpha} \PY{o}{=} \PY{l+m+mf}{0.5}\PY{p}{,} \PY{n}{kind} \PY{o}{=} \PY{l+s+s1}{\PYZsq{}}\PY{l+s+s1}{hist}\PY{l+s+s1}{\PYZsq{}}\PY{p}{,} \PY{n}{label} \PY{o}{=} \PY{l+s+s1}{\PYZsq{}}\PY{l+s+s1}{filtrated}\PY{l+s+s1}{\PYZsq{}}\PY{p}{,} \PY{n}{legend} \PY{o}{=} \PY{k+kc}{True}\PY{p}{)}
             \PY{n}{ax}\PY{o}{.}\PY{n}{axvline}\PY{p}{(}\PY{n}{data\PYZus{}origin}\PY{p}{[}\PY{n}{item}\PY{p}{]}\PY{o}{.}\PY{n}{mean}\PY{p}{(}\PY{p}{)}\PY{p}{,} \PY{n}{color} \PY{o}{=} \PY{l+s+s1}{\PYZsq{}}\PY{l+s+s1}{r}\PY{l+s+s1}{\PYZsq{}}\PY{p}{)}
             \PY{n}{ax}\PY{o}{.}\PY{n}{axvline}\PY{p}{(}\PY{n}{data\PYZus{}filtrated}\PY{p}{[}\PY{n}{item}\PY{p}{]}\PY{o}{.}\PY{n}{mean}\PY{p}{(}\PY{p}{)}\PY{p}{,} \PY{n}{color} \PY{o}{=} \PY{l+s+s1}{\PYZsq{}}\PY{l+s+s1}{b}\PY{l+s+s1}{\PYZsq{}}\PY{p}{)}
             \PY{n}{i} \PY{o}{+}\PY{o}{=} \PY{l+m+mi}{1}
         \PY{n}{plt}\PY{o}{.}\PY{n}{subplots\PYZus{}adjust}\PY{p}{(}\PY{n}{wspace} \PY{o}{=} \PY{l+m+mf}{0.3}\PY{p}{,} \PY{n}{hspace} \PY{o}{=} \PY{l+m+mf}{0.3}\PY{p}{)}
         
         \PY{c+c1}{\PYZsh{} 保存图像和处理后数据}
         \PY{n}{fig}\PY{o}{.}\PY{n}{savefig}\PY{p}{(}\PY{l+s+s1}{\PYZsq{}}\PY{l+s+s1}{./image/missing\PYZus{}data\PYZus{}delete.jpg}\PY{l+s+s1}{\PYZsq{}}\PY{p}{)}
         \PY{n}{data\PYZus{}filtrated}\PY{o}{.}\PY{n}{to\PYZus{}csv}\PY{p}{(}\PY{l+s+s1}{\PYZsq{}}\PY{l+s+s1}{./data\PYZus{}output/missing\PYZus{}data\PYZus{}delete.csv}\PY{l+s+s1}{\PYZsq{}}\PY{p}{,} \PY{n}{mode} \PY{o}{=} \PY{l+s+s1}{\PYZsq{}}\PY{l+s+s1}{w}\PY{l+s+s1}{\PYZsq{}}\PY{p}{,} \PY{n}{encoding}\PY{o}{=}\PY{l+s+s1}{\PYZsq{}}\PY{l+s+s1}{utf\PYZhy{}8}\PY{l+s+s1}{\PYZsq{}}\PY{p}{,} \PY{n}{index} \PY{o}{=} \PY{k+kc}{False}\PY{p}{,}\PY{n}{header} \PY{o}{=} \PY{k+kc}{False}\PY{p}{)}
\end{Verbatim}


    \begin{Verbatim}[commandchars=\\\{\}]
C:\textbackslash{}Users\textbackslash{}Adminstrator\textbackslash{}Anaconda3\textbackslash{}lib\textbackslash{}site-packages\textbackslash{}matplotlib\textbackslash{}cbook\textbackslash{}deprecation.py:107: MatplotlibDeprecationWarning: Adding an axes using the same arguments as a previous axes currently reuses the earlier instance.  In a future version, a new instance will always be created and returned.  Meanwhile, this warning can be suppressed, and the future behavior ensured, by passing a unique label to each axes instance.
  warnings.warn(message, mplDeprecation, stacklevel=1)

    \end{Verbatim}

    \begin{center}
    \adjustimage{max size={0.9\linewidth}{0.9\paperheight}}{output_21_1.png}
    \end{center}
    { \hspace*{\fill} \\}
    
    4.2 用最高频率值来填补缺失值

使用\textbf{\emph{value\_counts()}}函数统计原始数据中,出现频率最高的值,再用\textbf{\emph{fillna()}}函数将缺失值替换为最高频率值。

从折线图看出,处理后标称属性值不变;从直方图可以看出,数值属性的缺失值补全为高频值,均值基本保持不变。

    \begin{Verbatim}[commandchars=\\\{\}]
{\color{incolor}In [{\color{incolor}40}]:} \PY{c+c1}{\PYZsh{} 建立原始数据的拷贝}
         \PY{n}{data\PYZus{}filtrated} \PY{o}{=} \PY{n}{data\PYZus{}origin}\PY{o}{.}\PY{n}{copy}\PY{p}{(}\PY{p}{)}
         \PY{c+c1}{\PYZsh{} 对每一列数据,分别进行处理}
         \PY{k}{for} \PY{n}{item} \PY{o+ow}{in} \PY{n}{name\PYZus{}category}\PY{o}{+}\PY{n}{name\PYZus{}value}\PY{p}{:}
             \PY{c+c1}{\PYZsh{} 计算最高频率的值}
             \PY{n}{most\PYZus{}frequent\PYZus{}value} \PY{o}{=} \PY{n}{data\PYZus{}filtrated}\PY{p}{[}\PY{n}{item}\PY{p}{]}\PY{o}{.}\PY{n}{value\PYZus{}counts}\PY{p}{(}\PY{p}{)}\PY{o}{.}\PY{n}{idxmax}\PY{p}{(}\PY{p}{)}
             \PY{c+c1}{\PYZsh{} 替换缺失值}
             \PY{n}{data\PYZus{}filtrated}\PY{p}{[}\PY{n}{item}\PY{p}{]}\PY{o}{.}\PY{n}{fillna}\PY{p}{(}\PY{n}{value} \PY{o}{=} \PY{n}{most\PYZus{}frequent\PYZus{}value}\PY{p}{,} \PY{n}{inplace} \PY{o}{=} \PY{k+kc}{True}\PY{p}{)}
         
         \PY{c+c1}{\PYZsh{} 绘制可视化图}
         \PY{n}{fig} \PY{o}{=} \PY{n}{plt}\PY{o}{.}\PY{n}{figure}\PY{p}{(}\PY{n}{figsize} \PY{o}{=} \PY{p}{(}\PY{l+m+mi}{20}\PY{p}{,}\PY{l+m+mi}{15}\PY{p}{)}\PY{p}{)}
         
         \PY{n}{i} \PY{o}{=} \PY{l+m+mi}{1}
         \PY{c+c1}{\PYZsh{} 对标称属性,绘制折线图}
         \PY{k}{for} \PY{n}{item} \PY{o+ow}{in} \PY{n}{name\PYZus{}category}\PY{p}{:}
             \PY{n}{ax} \PY{o}{=} \PY{n}{fig}\PY{o}{.}\PY{n}{add\PYZus{}subplot}\PY{p}{(}\PY{l+m+mi}{4}\PY{p}{,} \PY{l+m+mi}{5}\PY{p}{,} \PY{n}{i}\PY{p}{)}
             \PY{n}{ax}\PY{o}{.}\PY{n}{set\PYZus{}title}\PY{p}{(}\PY{n}{item}\PY{p}{)}
             \PY{n}{pd}\PY{o}{.}\PY{n}{value\PYZus{}counts}\PY{p}{(}\PY{n}{data\PYZus{}origin}\PY{p}{[}\PY{n}{item}\PY{p}{]}\PY{o}{.}\PY{n}{values}\PY{p}{)}\PY{o}{.}\PY{n}{plot}\PY{p}{(}\PY{n}{ax} \PY{o}{=} \PY{n}{ax}\PY{p}{,} \PY{n}{marker} \PY{o}{=} \PY{l+s+s1}{\PYZsq{}}\PY{l+s+s1}{\PYZca{}}\PY{l+s+s1}{\PYZsq{}}\PY{p}{,} \PY{n}{label} \PY{o}{=} \PY{l+s+s1}{\PYZsq{}}\PY{l+s+s1}{origin}\PY{l+s+s1}{\PYZsq{}}\PY{p}{,} \PY{n}{legend} \PY{o}{=} \PY{k+kc}{True}\PY{p}{)}
             \PY{n}{pd}\PY{o}{.}\PY{n}{value\PYZus{}counts}\PY{p}{(}\PY{n}{data\PYZus{}filtrated}\PY{p}{[}\PY{n}{item}\PY{p}{]}\PY{o}{.}\PY{n}{values}\PY{p}{)}\PY{o}{.}\PY{n}{plot}\PY{p}{(}\PY{n}{ax} \PY{o}{=} \PY{n}{ax}\PY{p}{,} \PY{n}{marker} \PY{o}{=} \PY{l+s+s1}{\PYZsq{}}\PY{l+s+s1}{o}\PY{l+s+s1}{\PYZsq{}}\PY{p}{,} \PY{n}{label} \PY{o}{=} \PY{l+s+s1}{\PYZsq{}}\PY{l+s+s1}{filtrated}\PY{l+s+s1}{\PYZsq{}}\PY{p}{,} \PY{n}{legend} \PY{o}{=} \PY{k+kc}{True}\PY{p}{)}
             \PY{n}{i} \PY{o}{+}\PY{o}{=} \PY{l+m+mi}{1}    
         
         \PY{n}{i} \PY{o}{=} \PY{l+m+mi}{6}
         \PY{c+c1}{\PYZsh{} 对数值属性,绘制直方图}
         \PY{k}{for} \PY{n}{item} \PY{o+ow}{in} \PY{n}{name\PYZus{}value}\PY{p}{:}
             \PY{n}{ax} \PY{o}{=} \PY{n}{fig}\PY{o}{.}\PY{n}{add\PYZus{}subplot}\PY{p}{(}\PY{l+m+mi}{4}\PY{p}{,} \PY{l+m+mi}{5}\PY{p}{,} \PY{n}{i}\PY{p}{)}
             \PY{n}{ax}\PY{o}{.}\PY{n}{set\PYZus{}title}\PY{p}{(}\PY{n}{item}\PY{p}{)}
             \PY{n}{data\PYZus{}origin}\PY{p}{[}\PY{n}{item}\PY{p}{]}\PY{o}{.}\PY{n}{plot}\PY{p}{(}\PY{n}{ax} \PY{o}{=} \PY{n}{ax}\PY{p}{,} \PY{n}{alpha} \PY{o}{=} \PY{l+m+mf}{0.5}\PY{p}{,} \PY{n}{kind} \PY{o}{=} \PY{l+s+s1}{\PYZsq{}}\PY{l+s+s1}{hist}\PY{l+s+s1}{\PYZsq{}}\PY{p}{,} \PY{n}{label} \PY{o}{=} \PY{l+s+s1}{\PYZsq{}}\PY{l+s+s1}{origin}\PY{l+s+s1}{\PYZsq{}}\PY{p}{,} \PY{n}{legend} \PY{o}{=} \PY{k+kc}{True}\PY{p}{)}
             \PY{n}{data\PYZus{}filtrated}\PY{p}{[}\PY{n}{item}\PY{p}{]}\PY{o}{.}\PY{n}{plot}\PY{p}{(}\PY{n}{ax} \PY{o}{=} \PY{n}{ax}\PY{p}{,} \PY{n}{alpha} \PY{o}{=} \PY{l+m+mf}{0.5}\PY{p}{,} \PY{n}{kind} \PY{o}{=} \PY{l+s+s1}{\PYZsq{}}\PY{l+s+s1}{hist}\PY{l+s+s1}{\PYZsq{}}\PY{p}{,} \PY{n}{label} \PY{o}{=} \PY{l+s+s1}{\PYZsq{}}\PY{l+s+s1}{droped}\PY{l+s+s1}{\PYZsq{}}\PY{p}{,} \PY{n}{legend} \PY{o}{=} \PY{k+kc}{True}\PY{p}{)}
             \PY{n}{ax}\PY{o}{.}\PY{n}{axvline}\PY{p}{(}\PY{n}{data\PYZus{}origin}\PY{p}{[}\PY{n}{item}\PY{p}{]}\PY{o}{.}\PY{n}{mean}\PY{p}{(}\PY{p}{)}\PY{p}{,} \PY{n}{color} \PY{o}{=} \PY{l+s+s1}{\PYZsq{}}\PY{l+s+s1}{r}\PY{l+s+s1}{\PYZsq{}}\PY{p}{)}
             \PY{n}{ax}\PY{o}{.}\PY{n}{axvline}\PY{p}{(}\PY{n}{data\PYZus{}filtrated}\PY{p}{[}\PY{n}{item}\PY{p}{]}\PY{o}{.}\PY{n}{mean}\PY{p}{(}\PY{p}{)}\PY{p}{,} \PY{n}{color} \PY{o}{=} \PY{l+s+s1}{\PYZsq{}}\PY{l+s+s1}{b}\PY{l+s+s1}{\PYZsq{}}\PY{p}{)}
             \PY{n}{i} \PY{o}{+}\PY{o}{=} \PY{l+m+mi}{1}
         \PY{n}{plt}\PY{o}{.}\PY{n}{subplots\PYZus{}adjust}\PY{p}{(}\PY{n}{wspace} \PY{o}{=} \PY{l+m+mf}{0.3}\PY{p}{,} \PY{n}{hspace} \PY{o}{=} \PY{l+m+mf}{0.3}\PY{p}{)}
         
         \PY{c+c1}{\PYZsh{} 保存图像和处理后数据}
         \PY{n}{fig}\PY{o}{.}\PY{n}{savefig}\PY{p}{(}\PY{l+s+s1}{\PYZsq{}}\PY{l+s+s1}{./image/missing\PYZus{}data\PYZus{}most.jpg}\PY{l+s+s1}{\PYZsq{}}\PY{p}{)}
         \PY{n}{data\PYZus{}filtrated}\PY{o}{.}\PY{n}{to\PYZus{}csv}\PY{p}{(}\PY{l+s+s1}{\PYZsq{}}\PY{l+s+s1}{./data\PYZus{}output/missing\PYZus{}data\PYZus{}most.csv}\PY{l+s+s1}{\PYZsq{}}\PY{p}{,} \PY{n}{mode} \PY{o}{=} \PY{l+s+s1}{\PYZsq{}}\PY{l+s+s1}{w}\PY{l+s+s1}{\PYZsq{}}\PY{p}{,} \PY{n}{encoding}\PY{o}{=}\PY{l+s+s1}{\PYZsq{}}\PY{l+s+s1}{utf\PYZhy{}8}\PY{l+s+s1}{\PYZsq{}}\PY{p}{,} \PY{n}{index} \PY{o}{=} \PY{k+kc}{False}\PY{p}{,}\PY{n}{header} \PY{o}{=} \PY{k+kc}{False}\PY{p}{)}
\end{Verbatim}


    \begin{Verbatim}[commandchars=\\\{\}]
C:\textbackslash{}Users\textbackslash{}Adminstrator\textbackslash{}Anaconda3\textbackslash{}lib\textbackslash{}site-packages\textbackslash{}matplotlib\textbackslash{}cbook\textbackslash{}deprecation.py:107: MatplotlibDeprecationWarning: Adding an axes using the same arguments as a previous axes currently reuses the earlier instance.  In a future version, a new instance will always be created and returned.  Meanwhile, this warning can be suppressed, and the future behavior ensured, by passing a unique label to each axes instance.
  warnings.warn(message, mplDeprecation, stacklevel=1)

    \end{Verbatim}

    \begin{center}
    \adjustimage{max size={0.9\linewidth}{0.9\paperheight}}{output_23_1.png}
    \end{center}
    { \hspace*{\fill} \\}
    
    4.3 通过属性的相关关系来填补缺失值

使用pandas中Series的\textbf{\emph{interpolate()}}函数,对数值属性进行插值计算,并替换缺失值。

从直方图中可以看出,处理后的数据,添加了若干个值不同的值,并且均值变化不大。

    \begin{Verbatim}[commandchars=\\\{\}]
{\color{incolor}In [{\color{incolor}41}]:} \PY{c+c1}{\PYZsh{} 建立原始数据的拷贝}
         \PY{n}{data\PYZus{}filtrated} \PY{o}{=} \PY{n}{data\PYZus{}origin}\PY{o}{.}\PY{n}{copy}\PY{p}{(}\PY{p}{)}
         \PY{c+c1}{\PYZsh{} 对数值型属性的每一列,进行插值运算}
         \PY{k}{for} \PY{n}{item} \PY{o+ow}{in} \PY{n}{name\PYZus{}value}\PY{p}{:}
             \PY{n}{data\PYZus{}filtrated}\PY{p}{[}\PY{n}{item}\PY{p}{]}\PY{o}{.}\PY{n}{interpolate}\PY{p}{(}\PY{n}{inplace} \PY{o}{=} \PY{k+kc}{True}\PY{p}{)}
         
         \PY{c+c1}{\PYZsh{} 绘制可视化图}
         \PY{n}{fig} \PY{o}{=} \PY{n}{plt}\PY{o}{.}\PY{n}{figure}\PY{p}{(}\PY{n}{figsize} \PY{o}{=} \PY{p}{(}\PY{l+m+mi}{20}\PY{p}{,}\PY{l+m+mi}{15}\PY{p}{)}\PY{p}{)}
         
         \PY{n}{i} \PY{o}{=} \PY{l+m+mi}{1}
         \PY{c+c1}{\PYZsh{} 对标称属性,绘制折线图}
         \PY{k}{for} \PY{n}{item} \PY{o+ow}{in} \PY{n}{name\PYZus{}category}\PY{p}{:}
             \PY{n}{ax} \PY{o}{=} \PY{n}{fig}\PY{o}{.}\PY{n}{add\PYZus{}subplot}\PY{p}{(}\PY{l+m+mi}{4}\PY{p}{,} \PY{l+m+mi}{5}\PY{p}{,} \PY{n}{i}\PY{p}{)}
             \PY{n}{ax}\PY{o}{.}\PY{n}{set\PYZus{}title}\PY{p}{(}\PY{n}{item}\PY{p}{)}
             \PY{n}{pd}\PY{o}{.}\PY{n}{value\PYZus{}counts}\PY{p}{(}\PY{n}{data\PYZus{}origin}\PY{p}{[}\PY{n}{item}\PY{p}{]}\PY{o}{.}\PY{n}{values}\PY{p}{)}\PY{o}{.}\PY{n}{plot}\PY{p}{(}\PY{n}{ax} \PY{o}{=} \PY{n}{ax}\PY{p}{,} \PY{n}{marker} \PY{o}{=} \PY{l+s+s1}{\PYZsq{}}\PY{l+s+s1}{\PYZca{}}\PY{l+s+s1}{\PYZsq{}}\PY{p}{,} \PY{n}{label} \PY{o}{=} \PY{l+s+s1}{\PYZsq{}}\PY{l+s+s1}{origin}\PY{l+s+s1}{\PYZsq{}}\PY{p}{,} \PY{n}{legend} \PY{o}{=} \PY{k+kc}{True}\PY{p}{)}
             \PY{n}{pd}\PY{o}{.}\PY{n}{value\PYZus{}counts}\PY{p}{(}\PY{n}{data\PYZus{}filtrated}\PY{p}{[}\PY{n}{item}\PY{p}{]}\PY{o}{.}\PY{n}{values}\PY{p}{)}\PY{o}{.}\PY{n}{plot}\PY{p}{(}\PY{n}{ax} \PY{o}{=} \PY{n}{ax}\PY{p}{,} \PY{n}{marker} \PY{o}{=} \PY{l+s+s1}{\PYZsq{}}\PY{l+s+s1}{o}\PY{l+s+s1}{\PYZsq{}}\PY{p}{,} \PY{n}{label} \PY{o}{=} \PY{l+s+s1}{\PYZsq{}}\PY{l+s+s1}{filtrated}\PY{l+s+s1}{\PYZsq{}}\PY{p}{,} \PY{n}{legend} \PY{o}{=} \PY{k+kc}{True}\PY{p}{)}
             \PY{n}{i} \PY{o}{+}\PY{o}{=} \PY{l+m+mi}{1}   
             
         \PY{n}{i} \PY{o}{=} \PY{l+m+mi}{6}
         \PY{c+c1}{\PYZsh{} 对数值属性,绘制直方图}
         \PY{k}{for} \PY{n}{item} \PY{o+ow}{in} \PY{n}{name\PYZus{}value}\PY{p}{:}
             \PY{n}{ax} \PY{o}{=} \PY{n}{fig}\PY{o}{.}\PY{n}{add\PYZus{}subplot}\PY{p}{(}\PY{l+m+mi}{4}\PY{p}{,} \PY{l+m+mi}{5}\PY{p}{,} \PY{n}{i}\PY{p}{)}
             \PY{n}{ax}\PY{o}{.}\PY{n}{set\PYZus{}title}\PY{p}{(}\PY{n}{item}\PY{p}{)}
             \PY{n}{data\PYZus{}origin}\PY{p}{[}\PY{n}{item}\PY{p}{]}\PY{o}{.}\PY{n}{plot}\PY{p}{(}\PY{n}{ax} \PY{o}{=} \PY{n}{ax}\PY{p}{,} \PY{n}{alpha} \PY{o}{=} \PY{l+m+mf}{0.5}\PY{p}{,} \PY{n}{kind} \PY{o}{=} \PY{l+s+s1}{\PYZsq{}}\PY{l+s+s1}{hist}\PY{l+s+s1}{\PYZsq{}}\PY{p}{,} \PY{n}{label} \PY{o}{=} \PY{l+s+s1}{\PYZsq{}}\PY{l+s+s1}{origin}\PY{l+s+s1}{\PYZsq{}}\PY{p}{,} \PY{n}{legend} \PY{o}{=} \PY{k+kc}{True}\PY{p}{)}
             \PY{n}{data\PYZus{}filtrated}\PY{p}{[}\PY{n}{item}\PY{p}{]}\PY{o}{.}\PY{n}{plot}\PY{p}{(}\PY{n}{ax} \PY{o}{=} \PY{n}{ax}\PY{p}{,} \PY{n}{alpha} \PY{o}{=} \PY{l+m+mf}{0.5}\PY{p}{,} \PY{n}{kind} \PY{o}{=} \PY{l+s+s1}{\PYZsq{}}\PY{l+s+s1}{hist}\PY{l+s+s1}{\PYZsq{}}\PY{p}{,} \PY{n}{label} \PY{o}{=} \PY{l+s+s1}{\PYZsq{}}\PY{l+s+s1}{droped}\PY{l+s+s1}{\PYZsq{}}\PY{p}{,} \PY{n}{legend} \PY{o}{=} \PY{k+kc}{True}\PY{p}{)}
             \PY{n}{ax}\PY{o}{.}\PY{n}{axvline}\PY{p}{(}\PY{n}{data\PYZus{}origin}\PY{p}{[}\PY{n}{item}\PY{p}{]}\PY{o}{.}\PY{n}{mean}\PY{p}{(}\PY{p}{)}\PY{p}{,} \PY{n}{color} \PY{o}{=} \PY{l+s+s1}{\PYZsq{}}\PY{l+s+s1}{r}\PY{l+s+s1}{\PYZsq{}}\PY{p}{)}
             \PY{n}{ax}\PY{o}{.}\PY{n}{axvline}\PY{p}{(}\PY{n}{data\PYZus{}filtrated}\PY{p}{[}\PY{n}{item}\PY{p}{]}\PY{o}{.}\PY{n}{mean}\PY{p}{(}\PY{p}{)}\PY{p}{,} \PY{n}{color} \PY{o}{=} \PY{l+s+s1}{\PYZsq{}}\PY{l+s+s1}{b}\PY{l+s+s1}{\PYZsq{}}\PY{p}{)}
             \PY{n}{i} \PY{o}{+}\PY{o}{=} \PY{l+m+mi}{1}
         \PY{n}{plt}\PY{o}{.}\PY{n}{subplots\PYZus{}adjust}\PY{p}{(}\PY{n}{wspace} \PY{o}{=} \PY{l+m+mf}{0.3}\PY{p}{,} \PY{n}{hspace} \PY{o}{=} \PY{l+m+mf}{0.3}\PY{p}{)}
         
         \PY{c+c1}{\PYZsh{} 保存图像和处理后数据}
         \PY{n}{fig}\PY{o}{.}\PY{n}{savefig}\PY{p}{(}\PY{l+s+s1}{\PYZsq{}}\PY{l+s+s1}{./image/missing\PYZus{}data\PYZus{}corelation.jpg}\PY{l+s+s1}{\PYZsq{}}\PY{p}{)}
         \PY{n}{data\PYZus{}filtrated}\PY{o}{.}\PY{n}{to\PYZus{}csv}\PY{p}{(}\PY{l+s+s1}{\PYZsq{}}\PY{l+s+s1}{./data\PYZus{}output/missing\PYZus{}data\PYZus{}corelation.csv}\PY{l+s+s1}{\PYZsq{}}\PY{p}{,} \PY{n}{mode} \PY{o}{=} \PY{l+s+s1}{\PYZsq{}}\PY{l+s+s1}{w}\PY{l+s+s1}{\PYZsq{}}\PY{p}{,} \PY{n}{encoding}\PY{o}{=}\PY{l+s+s1}{\PYZsq{}}\PY{l+s+s1}{utf\PYZhy{}8}\PY{l+s+s1}{\PYZsq{}}\PY{p}{,} \PY{n}{index} \PY{o}{=} \PY{k+kc}{False}\PY{p}{,}\PY{n}{header} \PY{o}{=} \PY{k+kc}{False}\PY{p}{)}
\end{Verbatim}


    \begin{Verbatim}[commandchars=\\\{\}]
C:\textbackslash{}Users\textbackslash{}Adminstrator\textbackslash{}Anaconda3\textbackslash{}lib\textbackslash{}site-packages\textbackslash{}matplotlib\textbackslash{}cbook\textbackslash{}deprecation.py:107: MatplotlibDeprecationWarning: Adding an axes using the same arguments as a previous axes currently reuses the earlier instance.  In a future version, a new instance will always be created and returned.  Meanwhile, this warning can be suppressed, and the future behavior ensured, by passing a unique label to each axes instance.
  warnings.warn(message, mplDeprecation, stacklevel=1)

    \end{Verbatim}

    \begin{center}
    \adjustimage{max size={0.9\linewidth}{0.9\paperheight}}{output_25_1.png}
    \end{center}
    { \hspace*{\fill} \\}
    
    4.4 通过数据对象之间的相似性来填补缺失值

首先将缺失值设为0,对数据集进行正则化。然后对每两条数据进行差异性计算(分值越高差异性越大)。计算标准为:标称数据不相同记为1分,数值数据差异性分数为数据之间的差值。在处理缺失值时,找到和该条数据对象差异性最小(分数最低)的对象,将最相似的数据条目中对应属性的值替换缺失值。

    \begin{Verbatim}[commandchars=\\\{\}]
{\color{incolor}In [{\color{incolor}43}]:} \PY{c+c1}{\PYZsh{} 建立原始数据的拷贝,用于正则化处理}
         \PY{n}{data\PYZus{}norm} \PY{o}{=} \PY{n}{data\PYZus{}origin}\PY{o}{.}\PY{n}{copy}\PY{p}{(}\PY{p}{)}
         \PY{c+c1}{\PYZsh{} 将数值属性的缺失值替换为0}
         \PY{n}{data\PYZus{}norm}\PY{p}{[}\PY{n}{name\PYZus{}value}\PY{p}{]} \PY{o}{=} \PY{n}{data\PYZus{}norm}\PY{p}{[}\PY{n}{name\PYZus{}value}\PY{p}{]}\PY{o}{.}\PY{n}{fillna}\PY{p}{(}\PY{l+m+mi}{0}\PY{p}{)}
         \PY{c+c1}{\PYZsh{} 对数据进行正则化}
         \PY{n}{data\PYZus{}norm}\PY{p}{[}\PY{n}{name\PYZus{}value}\PY{p}{]} \PY{o}{=} \PY{n}{data\PYZus{}norm}\PY{p}{[}\PY{n}{name\PYZus{}value}\PY{p}{]}\PY{o}{.}\PY{n}{apply}\PY{p}{(}\PY{k}{lambda} \PY{n}{x} \PY{p}{:} \PY{p}{(}\PY{n}{x} \PY{o}{\PYZhy{}} \PY{n}{np}\PY{o}{.}\PY{n}{mean}\PY{p}{(}\PY{n}{x}\PY{p}{)}\PY{p}{)} \PY{o}{/} \PY{p}{(}\PY{n}{np}\PY{o}{.}\PY{n}{max}\PY{p}{(}\PY{n}{x}\PY{p}{)} \PY{o}{\PYZhy{}} \PY{n}{np}\PY{o}{.}\PY{n}{min}\PY{p}{(}\PY{n}{x}\PY{p}{)}\PY{p}{)}\PY{p}{)}
         
         \PY{c+c1}{\PYZsh{} 构造分数表}
         \PY{n}{score} \PY{o}{=} \PY{p}{\PYZob{}}\PY{p}{\PYZcb{}}
         \PY{n}{range\PYZus{}length} \PY{o}{=} \PY{n+nb}{len}\PY{p}{(}\PY{n}{data\PYZus{}origin}\PY{p}{)}
         \PY{k}{for} \PY{n}{i} \PY{o+ow}{in} \PY{n+nb}{range}\PY{p}{(}\PY{l+m+mi}{0}\PY{p}{,} \PY{n}{range\PYZus{}length}\PY{p}{)}\PY{p}{:}
             \PY{n}{score}\PY{p}{[}\PY{n}{i}\PY{p}{]} \PY{o}{=} \PY{p}{\PYZob{}}\PY{p}{\PYZcb{}}
             \PY{k}{for} \PY{n}{j} \PY{o+ow}{in} \PY{n+nb}{range}\PY{p}{(}\PY{l+m+mi}{0}\PY{p}{,} \PY{n}{range\PYZus{}length}\PY{p}{)}\PY{p}{:}
                 \PY{n}{score}\PY{p}{[}\PY{n}{i}\PY{p}{]}\PY{p}{[}\PY{n}{j}\PY{p}{]} \PY{o}{=} \PY{l+m+mi}{0}    
         
         \PY{c+c1}{\PYZsh{} 在处理后的数据中,对每两条数据条目计算差异性得分,分值越高差异性越大}
         \PY{k}{for} \PY{n}{i} \PY{o+ow}{in} \PY{n+nb}{range}\PY{p}{(}\PY{l+m+mi}{0}\PY{p}{,} \PY{n}{range\PYZus{}length}\PY{p}{)}\PY{p}{:}
             \PY{k}{for} \PY{n}{j} \PY{o+ow}{in} \PY{n+nb}{range}\PY{p}{(}\PY{n}{i}\PY{p}{,} \PY{n}{range\PYZus{}length}\PY{p}{)}\PY{p}{:}
                 \PY{k}{for} \PY{n}{item} \PY{o+ow}{in} \PY{n}{name\PYZus{}category}\PY{p}{:}
                     \PY{k}{if} \PY{n}{data\PYZus{}norm}\PY{o}{.}\PY{n}{iloc}\PY{p}{[}\PY{n}{i}\PY{p}{]}\PY{p}{[}\PY{n}{item}\PY{p}{]} \PY{o}{!=} \PY{n}{data\PYZus{}norm}\PY{o}{.}\PY{n}{iloc}\PY{p}{[}\PY{n}{j}\PY{p}{]}\PY{p}{[}\PY{n}{item}\PY{p}{]}\PY{p}{:}
                         \PY{n}{score}\PY{p}{[}\PY{n}{i}\PY{p}{]}\PY{p}{[}\PY{n}{j}\PY{p}{]} \PY{o}{+}\PY{o}{=} \PY{l+m+mi}{1}
                 \PY{k}{for} \PY{n}{item} \PY{o+ow}{in} \PY{n}{name\PYZus{}value}\PY{p}{:}
                     \PY{n}{temp} \PY{o}{=} \PY{n+nb}{abs}\PY{p}{(}\PY{n}{data\PYZus{}norm}\PY{o}{.}\PY{n}{iloc}\PY{p}{[}\PY{n}{i}\PY{p}{]}\PY{p}{[}\PY{n}{item}\PY{p}{]} \PY{o}{\PYZhy{}} \PY{n}{data\PYZus{}norm}\PY{o}{.}\PY{n}{iloc}\PY{p}{[}\PY{n}{j}\PY{p}{]}\PY{p}{[}\PY{n}{item}\PY{p}{]}\PY{p}{)}
                     \PY{n}{score}\PY{p}{[}\PY{n}{i}\PY{p}{]}\PY{p}{[}\PY{n}{j}\PY{p}{]} \PY{o}{+}\PY{o}{=} \PY{n}{temp}
                 \PY{n}{score}\PY{p}{[}\PY{n}{j}\PY{p}{]}\PY{p}{[}\PY{n}{i}\PY{p}{]} \PY{o}{=} \PY{n}{score}\PY{p}{[}\PY{n}{i}\PY{p}{]}\PY{p}{[}\PY{n}{j}\PY{p}{]}
         
         \PY{c+c1}{\PYZsh{} 建立原始数据的拷贝}
         \PY{n}{data\PYZus{}filtrated} \PY{o}{=} \PY{n}{data\PYZus{}origin}\PY{o}{.}\PY{n}{copy}\PY{p}{(}\PY{p}{)}
         
         \PY{c+c1}{\PYZsh{} 对有缺失值的条目,用和它相似度最高(得分最低)的数据条目中对应属性的值替换}
         \PY{k}{for} \PY{n}{index} \PY{o+ow}{in} \PY{n}{nan\PYZus{}list}\PY{p}{:}
             \PY{n}{best\PYZus{}friend} \PY{o}{=} \PY{n+nb}{sorted}\PY{p}{(}\PY{n}{score}\PY{p}{[}\PY{n}{index}\PY{p}{]}\PY{o}{.}\PY{n}{items}\PY{p}{(}\PY{p}{)}\PY{p}{,} \PY{n}{key}\PY{o}{=}\PY{n}{operator}\PY{o}{.}\PY{n}{itemgetter}\PY{p}{(}\PY{l+m+mi}{1}\PY{p}{)}\PY{p}{,} \PY{n}{reverse} \PY{o}{=} \PY{k+kc}{False}\PY{p}{)}\PY{p}{[}\PY{l+m+mi}{1}\PY{p}{]}\PY{p}{[}\PY{l+m+mi}{0}\PY{p}{]}
             \PY{k}{for} \PY{n}{item} \PY{o+ow}{in} \PY{n}{name\PYZus{}value}\PY{p}{:}
                 \PY{k}{if} \PY{n}{pd}\PY{o}{.}\PY{n}{isnull}\PY{p}{(}\PY{n}{data\PYZus{}filtrated}\PY{o}{.}\PY{n}{iloc}\PY{p}{[}\PY{n}{index}\PY{p}{]}\PY{p}{[}\PY{n}{item}\PY{p}{]}\PY{p}{)}\PY{p}{:}
                     \PY{k}{if} \PY{n}{pd}\PY{o}{.}\PY{n}{isnull}\PY{p}{(}\PY{n}{data\PYZus{}origin}\PY{o}{.}\PY{n}{iloc}\PY{p}{[}\PY{n}{best\PYZus{}friend}\PY{p}{]}\PY{p}{[}\PY{n}{item}\PY{p}{]}\PY{p}{)}\PY{p}{:}
                         \PY{n}{data\PYZus{}filtrated}\PY{o}{.}\PY{n}{ix}\PY{p}{[}\PY{n}{index}\PY{p}{,} \PY{n}{item}\PY{p}{]} \PY{o}{=} \PY{n}{data\PYZus{}origin}\PY{p}{[}\PY{n}{item}\PY{p}{]}\PY{o}{.}\PY{n}{value\PYZus{}counts}\PY{p}{(}\PY{p}{)}\PY{o}{.}\PY{n}{idxmax}\PY{p}{(}\PY{p}{)}
                     \PY{k}{else}\PY{p}{:}
                         \PY{n}{data\PYZus{}filtrated}\PY{o}{.}\PY{n}{ix}\PY{p}{[}\PY{n}{index}\PY{p}{,} \PY{n}{item}\PY{p}{]} \PY{o}{=} \PY{n}{data\PYZus{}origin}\PY{o}{.}\PY{n}{iloc}\PY{p}{[}\PY{n}{best\PYZus{}friend}\PY{p}{]}\PY{p}{[}\PY{n}{item}\PY{p}{]}
         
         \PY{c+c1}{\PYZsh{} 绘制可视化图}
         \PY{n}{fig} \PY{o}{=} \PY{n}{plt}\PY{o}{.}\PY{n}{figure}\PY{p}{(}\PY{n}{figsize} \PY{o}{=} \PY{p}{(}\PY{l+m+mi}{20}\PY{p}{,}\PY{l+m+mi}{15}\PY{p}{)}\PY{p}{)}
         
         \PY{n}{i} \PY{o}{=} \PY{l+m+mi}{1}
         \PY{c+c1}{\PYZsh{} 对标称属性,绘制折线图}
         \PY{k}{for} \PY{n}{item} \PY{o+ow}{in} \PY{n}{name\PYZus{}category}\PY{p}{:}
             \PY{n}{ax} \PY{o}{=} \PY{n}{fig}\PY{o}{.}\PY{n}{add\PYZus{}subplot}\PY{p}{(}\PY{l+m+mi}{4}\PY{p}{,} \PY{l+m+mi}{5}\PY{p}{,} \PY{n}{i}\PY{p}{)}
             \PY{n}{ax}\PY{o}{.}\PY{n}{set\PYZus{}title}\PY{p}{(}\PY{n}{item}\PY{p}{)}
             \PY{n}{pd}\PY{o}{.}\PY{n}{value\PYZus{}counts}\PY{p}{(}\PY{n}{data\PYZus{}origin}\PY{p}{[}\PY{n}{item}\PY{p}{]}\PY{o}{.}\PY{n}{values}\PY{p}{)}\PY{o}{.}\PY{n}{plot}\PY{p}{(}\PY{n}{ax} \PY{o}{=} \PY{n}{ax}\PY{p}{,} \PY{n}{marker} \PY{o}{=} \PY{l+s+s1}{\PYZsq{}}\PY{l+s+s1}{\PYZca{}}\PY{l+s+s1}{\PYZsq{}}\PY{p}{,} \PY{n}{label} \PY{o}{=} \PY{l+s+s1}{\PYZsq{}}\PY{l+s+s1}{origin}\PY{l+s+s1}{\PYZsq{}}\PY{p}{,} \PY{n}{legend} \PY{o}{=} \PY{k+kc}{True}\PY{p}{)}
             \PY{n}{pd}\PY{o}{.}\PY{n}{value\PYZus{}counts}\PY{p}{(}\PY{n}{data\PYZus{}filtrated}\PY{p}{[}\PY{n}{item}\PY{p}{]}\PY{o}{.}\PY{n}{values}\PY{p}{)}\PY{o}{.}\PY{n}{plot}\PY{p}{(}\PY{n}{ax} \PY{o}{=} \PY{n}{ax}\PY{p}{,} \PY{n}{marker} \PY{o}{=} \PY{l+s+s1}{\PYZsq{}}\PY{l+s+s1}{o}\PY{l+s+s1}{\PYZsq{}}\PY{p}{,} \PY{n}{label} \PY{o}{=} \PY{l+s+s1}{\PYZsq{}}\PY{l+s+s1}{filtrated}\PY{l+s+s1}{\PYZsq{}}\PY{p}{,} \PY{n}{legend} \PY{o}{=} \PY{k+kc}{True}\PY{p}{)}
             \PY{n}{i} \PY{o}{+}\PY{o}{=} \PY{l+m+mi}{1}   
             
         \PY{n}{i} \PY{o}{=} \PY{l+m+mi}{6}
         \PY{c+c1}{\PYZsh{} 对数值属性,绘制直方图}
         \PY{k}{for} \PY{n}{item} \PY{o+ow}{in} \PY{n}{name\PYZus{}value}\PY{p}{:}
             \PY{n}{ax} \PY{o}{=} \PY{n}{fig}\PY{o}{.}\PY{n}{add\PYZus{}subplot}\PY{p}{(}\PY{l+m+mi}{4}\PY{p}{,} \PY{l+m+mi}{5}\PY{p}{,} \PY{n}{i}\PY{p}{)}
             \PY{n}{ax}\PY{o}{.}\PY{n}{set\PYZus{}title}\PY{p}{(}\PY{n}{item}\PY{p}{)}
             \PY{n}{data\PYZus{}origin}\PY{p}{[}\PY{n}{item}\PY{p}{]}\PY{o}{.}\PY{n}{plot}\PY{p}{(}\PY{n}{ax} \PY{o}{=} \PY{n}{ax}\PY{p}{,} \PY{n}{alpha} \PY{o}{=} \PY{l+m+mf}{0.5}\PY{p}{,} \PY{n}{kind} \PY{o}{=} \PY{l+s+s1}{\PYZsq{}}\PY{l+s+s1}{hist}\PY{l+s+s1}{\PYZsq{}}\PY{p}{,} \PY{n}{label} \PY{o}{=} \PY{l+s+s1}{\PYZsq{}}\PY{l+s+s1}{origin}\PY{l+s+s1}{\PYZsq{}}\PY{p}{,} \PY{n}{legend} \PY{o}{=} \PY{k+kc}{True}\PY{p}{)}
             \PY{n}{data\PYZus{}filtrated}\PY{p}{[}\PY{n}{item}\PY{p}{]}\PY{o}{.}\PY{n}{plot}\PY{p}{(}\PY{n}{ax} \PY{o}{=} \PY{n}{ax}\PY{p}{,} \PY{n}{alpha} \PY{o}{=} \PY{l+m+mf}{0.5}\PY{p}{,} \PY{n}{kind} \PY{o}{=} \PY{l+s+s1}{\PYZsq{}}\PY{l+s+s1}{hist}\PY{l+s+s1}{\PYZsq{}}\PY{p}{,} \PY{n}{label} \PY{o}{=} \PY{l+s+s1}{\PYZsq{}}\PY{l+s+s1}{droped}\PY{l+s+s1}{\PYZsq{}}\PY{p}{,} \PY{n}{legend} \PY{o}{=} \PY{k+kc}{True}\PY{p}{)}
             \PY{n}{ax}\PY{o}{.}\PY{n}{axvline}\PY{p}{(}\PY{n}{data\PYZus{}origin}\PY{p}{[}\PY{n}{item}\PY{p}{]}\PY{o}{.}\PY{n}{mean}\PY{p}{(}\PY{p}{)}\PY{p}{,} \PY{n}{color} \PY{o}{=} \PY{l+s+s1}{\PYZsq{}}\PY{l+s+s1}{r}\PY{l+s+s1}{\PYZsq{}}\PY{p}{)}
             \PY{n}{ax}\PY{o}{.}\PY{n}{axvline}\PY{p}{(}\PY{n}{data\PYZus{}filtrated}\PY{p}{[}\PY{n}{item}\PY{p}{]}\PY{o}{.}\PY{n}{mean}\PY{p}{(}\PY{p}{)}\PY{p}{,} \PY{n}{color} \PY{o}{=} \PY{l+s+s1}{\PYZsq{}}\PY{l+s+s1}{b}\PY{l+s+s1}{\PYZsq{}}\PY{p}{)}
             \PY{n}{i} \PY{o}{+}\PY{o}{=} \PY{l+m+mi}{1}
         \PY{n}{plt}\PY{o}{.}\PY{n}{subplots\PYZus{}adjust}\PY{p}{(}\PY{n}{wspace} \PY{o}{=} \PY{l+m+mf}{0.3}\PY{p}{,} \PY{n}{hspace} \PY{o}{=} \PY{l+m+mf}{0.3}\PY{p}{)}
         
         \PY{c+c1}{\PYZsh{} 保存图像和处理后数据}
         \PY{n}{fig}\PY{o}{.}\PY{n}{savefig}\PY{p}{(}\PY{l+s+s1}{\PYZsq{}}\PY{l+s+s1}{./image/missing\PYZus{}data\PYZus{}similarity.jpg}\PY{l+s+s1}{\PYZsq{}}\PY{p}{)}
         \PY{n}{data\PYZus{}filtrated}\PY{o}{.}\PY{n}{to\PYZus{}csv}\PY{p}{(}\PY{l+s+s1}{\PYZsq{}}\PY{l+s+s1}{./data\PYZus{}output/missing\PYZus{}data\PYZus{}similarity.csv}\PY{l+s+s1}{\PYZsq{}}\PY{p}{,} \PY{n}{mode} \PY{o}{=} \PY{l+s+s1}{\PYZsq{}}\PY{l+s+s1}{w}\PY{l+s+s1}{\PYZsq{}}\PY{p}{,} \PY{n}{encoding}\PY{o}{=}\PY{l+s+s1}{\PYZsq{}}\PY{l+s+s1}{utf\PYZhy{}8}\PY{l+s+s1}{\PYZsq{}}\PY{p}{,} \PY{n}{index} \PY{o}{=} \PY{k+kc}{False}\PY{p}{,}\PY{n}{header} \PY{o}{=} \PY{k+kc}{False}\PY{p}{)}
\end{Verbatim}


    \begin{Verbatim}[commandchars=\\\{\}]

        ---------------------------------------------------------------------------

        MemoryError                               Traceback (most recent call last)

        <ipython-input-43-703fc2eb94f1> in <module>()
         12     score[i] = \{\}
         13     for j in range(0, range\_length):
    ---> 14         score[i][j] = 0
         15 
         16 \# 在处理后的数据中,对每两条数据条目计算差异性得分,分值越高差异性越大
    

        MemoryError: 

    \end{Verbatim}


    % Add a bibliography block to the postdoc
    
    
    
    \end{document}
